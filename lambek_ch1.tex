\documentclass[lambek.tex]{subfiles}


\begin{document}



\section{Gelfand duality}
Let $X$ be a  space, and consider the ring of real-valued functions,
\[
 \mathcal{C}(X)\ =\ \mathsf{Top}(X, \mathbb{R}).
 \]
 This  is a (contravariant) functor from ``geometry'' to ``algebra'':
 \[
 \mathcal{C} : \mathsf{Top}^\mathsf{op} \to \mathsf{CRng}.
 \]
%
It is full and faithful if we restrict to compact Hausdorf spaces and bounded continuous functions  $\mathcal{C}^*(X)$:
  \[
 \mathcal{C}^* : \mathsf{KHaus}^\mathsf{op} \hookrightarrow \mathsf{CRng}.
 \]
 \begin{theorem}[Gelfand duality]
 $\mathsf{KHaus}$ is dual to the category of all commutative rings of the form  $\mathcal{C}^*(X)$ and ring homomorphisms between them.
 \end{theorem}
 \medskip
 
 It then requires some work to  determine \emph{which} rings are of the form $\mathcal{C}^*(X)$!  
 They are called \emph{$C^*$-algebras}.

%%\end{frame}
%%%%%%%%%%%%%%%%%%%%%%%%%%%%%%%%%%%%%%%%%%%%%%%%%%%%%%%%%
%%%%%%%%%%%%%%%%%%%%%%%%%%%%%%%%%%%%%%%%%%%%%%%%%%%%%%%%%%
%\section{Gelfand duality}
When can we recover the space $X$ from its ring of functions $\mathcal{C}^*(X)$?

\begin{itemize}
\item The points  $x\in X$ determine maximal ideals in the ring $\mathcal{C}^*(X)$,
\[
M_x = \{ f : X\to \mathbb{R}\ |\ f(x) = 0\ \}
\]
\item The (Zariski) topology on the set $\mathsf{MaxIdl}(A)$ in any ring $A$ has a basis of open sets of the form:
\[
B_a\ =\ \{M \in X\ |\ a\notin M\ \}\, ,\qquad a\in A.
\]
\item If $A$ is a \emph{$C^*$-algebra}, then this specification will determine a compact Hausdorf space $X = \mathsf{MaxIdl}(A)$ such that $A\cong \mathcal{C}(X)$.
\end{itemize}

 \begin{theorem}[Gelfand-Stone-Naimark]
 $\mathsf{KHaus}$ is dual to the category of $C^*$-algebras,
 \[
\mathsf{KHaus}^\mathsf{op}\ \simeq\ C^*\mathsf{Alg}.
 \]
 \end{theorem}

%%\end{frame}
%%%%%%%%%%%%%%%%%%%%%%%%%%%%%%%%%%%%%%%%%%%%%%%%%%%%%%%%%
%%%%%%%%%%%%%%%%%%%%%%%%%%%%%%%%%%%%%%%%%%%%%%%%%%%%%%%%%
\section{Grothendieck's sheaf representation for commutative rings}

Grothendieck extended this duality from $C^*$-algebras to \emph{all} commutative rings, by generalizing on the ``geometric'' side from spaces to \emph{(affine) schemes},
 \[
\mathsf{Scheme}_\mathsf{aff}^\mathsf{op}\ \simeq\ \mathsf{CRng}.
 \]

The essential change was to generalize the ``ring of values'' from the constant ring $\mathbb{R}$ to a ring $\mathcal{R}$ that ``varies continuously over the space $X$'', i.e.\ a \emph{sheaf of rings}.   
\medskip

The various rings $\mathcal{R}_x$ generalize the \emph{local rings} of real-valued functions that vanish at $x\in X$.
\medskip

This allows \emph{every} commutative ring $A$ to be seen as a ring of continuous functions on a suitable space $X_A$, with values in a suitable sheaf of rings $\mathcal{R}$ on $X_A$.

%%\end{frame}
%%%%%%%%%%%%%%%%%%%%%%%%%%%%%%%%%%%%%%%%%%%%%%%%%%%%%%%%%
%%%%%%%%%%%%%%%%%%%%%%%%%%%%%%%%%%%%%%%%%%%%%%%%%%%%%%%%%
%\section{Grothendieck's sheaf representation for commutative rings}

\begin{definition} A  ring (commutative, with unit $1\neq 0$) is called \emph{local} if it has a unique maximal ideal.
Equivalently: $$x+y\ \text{is a unit}\quad\text{implies}\quad x\ \text{is a unit}\ \text{or}\ y\ \text{is a unit}.$$
\end{definition}
%
\begin{theorem}[Grothendieck]
Let $A$ be a ring.  There is a space $X$ with a sheaf of rings $\mathcal{R}$ such that:
\begin{enumerate}
\item for every $p\in X$, the stalk $\mathcal{R}_p$ is a local ring, 
\item for the ring of global sections, we have: $\Gamma(\mathcal{R}) \cong A$.
\end{enumerate}
Thus \emph{every ring is isomorphic to the ring of global sections of a sheaf of local rings}.
\end{theorem}

%%\end{frame}
%%%%%%%%%%%%%%%%%%%%%%%%%%%%%%%%%%%%%%%%%%%%%%%%%%%%%%%%%
%%%%%%%%%%%%%%%%%%%%%%%%%%%%%%%%%%%%%%%%%%%%%%%%%%%%%%%%%
%\section{Grothendieck's sheaf representation for commutative rings}

The \myemph{space} $X$ is the \emph{prime spectrum} $\mathsf{Spec}(A)$:
 \begin{enumerate}
\item points $p\in \mathsf{Spec}(A)$ are prime ideals $p\subseteq A$,
\item the topology has basic opens of the following form, for all $f\in A$:
 $$B_f = \{ p\in \mathsf{Spec}(A)\ |\ f\not\in p \}.$$
\end{enumerate}
The \myemph{structure sheaf} $\mathcal {R}$ is determined by ``localizing'' $A$ at $f$,
\[
\mathcal{R}(B_f) = [f]^{-1}A
\]
where $A \rightarrow [f]^{-1}A$ freely inverts all of the elements $f, f^2, f^3, \dots$.
\medskip

The \myemph{stalk} $\mathcal {R}_p$ is then the localization of $A$ at $p$,
\[
\mathcal{R}_p = S^{-1}A, 
\]
where $S = A\setminus p$.

%%\end{frame}
%%%%%%%%%%%%%%%%%%%%%%%%%%%%%%%%%%%%%%%%%%%%%%%%%%%%%%%%%
%%%%%%%%%%%%%%%%%%%%%%%%%%%%%%%%%%%%%%%%%%%%%%%%%%%%%%%%%
%\section{Grothendieck's sheaf representation for commutative rings}

The \myemph{affine scheme} $(\mathsf{Spec}(A), \mathcal {O})$ represents $A$ as a ``ring of continuous functions'' 
\[
f : \mathsf{Spec}(A) \to \mathcal {R}\,,
\]
\myemph{except} that the ring $\mathcal{R}$ is itself ``varying continuously over the space $\mathsf{Spec}(A)$'' (i.e.\ it is a sheaf).  
\medskip

The local ring $\mathcal{R}_p$ has a \myemph{unique maximal ideal}, consisting of ``those functions  $f : \mathsf{Spec}(A) \to \mathcal{R}$ that vanish at $p$''.
\medskip

It is a ``representation'' of $A$ because there is always an injective homomorphism
\[
A \cong \Gamma(\mathcal{R}) \rightarrowtail \prod_{p}\mathcal{R}_p \,.
\]

\begin{corollary}[Sub-direct-product representation]
Every ring $A$ is isomorphic to a \myemph{sub}ring of a \myemph{direct product} of local rings.
\end{corollary}

%%\end{frame}
%%%%%%%%%%%%%%%%%%%%%%%%%%%%%%%%%%%%%%%%%%%%%%%%%%%%%%%%%
%%%%%%%%%%%%%%%%%%%%%%%%%%%%%%%%%%%%%%%%%%%%%%%%%%%%%%%%%
\section{Lambek-Moerdijk sheaf representation for toposes}

\begin{definition} A  (small, elementary)  topos is called \emph{sublocal} if its subterminal lattice $\mathsf{Sub}(1)$ has a unique maximal ideal.
Equivalently, for $x,y\in \mathsf{Sub}(1)$: 
\[
x\vee y = 1\quad\text{implies}\quad x=1\ \text{or}\ y=1\,.
\]
\end{definition}
%
\begin{theorem}[Lambek-Moerdijk 1982]
Let $\E$ be a topos.  There is a space $X$ with a sheaf of toposes $\tilde\E$ such that:
\begin{enumerate}
\item for every $p\in X$, the stalk $\tilde\E_p$ is a sublocal topos, 
\item for the topos of global sections, we have: $\Gamma(\tilde\E) \cong \E$.
\end{enumerate}
Thus \emph{every topos is isomorphic to the topos of global sections of a sheaf of sublocal toposes}.
\end{theorem}

%%\end{frame}
%%%%%%%%%%%%%%%%%%%%%%%%%%%%%%%%%%%%%%%%%%%%%%%%%%%%%%%%%
%%%%%%%%%%%%%%%%%%%%%%%%%%%%%%%%%%%%%%%%%%%%%%%%%%%%%%%%%
%\section{Lambek-Moerdijk sheaf representation for toposes}

The \myemph{space} $X$ is the so-called \emph{(sub)spectrum of the topos},  $\mathsf{Spec}(\E)$.\\  
\medskip

It is the prime spectrum of the distributive lattice $\mathsf{Sub}(1)$:
 \begin{enumerate}
\item the points $P\in \mathsf{Spec}(\E)$ are prime ideals $P\subseteq \mathsf{Sub}(1)$,
\item the basic opens have the following form, for all $q\in\mathsf{Sub}(1)$:
$$B_q = \{ P\in \mathsf{Spec}(\E)\ |\ q\not\in P \}.$$
\end{enumerate}

The lattice of all open sets of $\mathsf{Spec}(\E)$ is isomorphic to the ideal completion of $\mathsf{Sub}(1)$,
$$O(\mathsf{Spec}(\E)) = \mathsf{Idl}(\mathsf{Sub}(1))\,.$$

%Equivalently, we could have set:
%\begin{enumerate}
%\item points $F\in Spec(\E)$ are prime filters $F\subseteq \mathsf{Sub}(1)$,
%\item the topology has basic opens of the following form, for $q\in\mathsf{Sub}(1)$:
%$$D_q = \{ F\in Spec(\E)\ \ |\ q\in F \}.$$
%\end{enumerate}
%

%%\end{frame}
%%%%%%%%%%%%%%%%%%%%%%%%%%%%%%%%%%%%%%%%%%%%%%%%%%%%%%%%%
%%%%%%%%%%%%%%%%%%%%%%%%%%%%%%%%%%%%%%%%%%%%%%%%%%%%%%%%%
%\section{Lambek-Moerdijk sheaf representation for toposes}


The \myemph{structure sheaf} $\tilde{\E}$ is determined by ``slicing'' $\E$ at $q \in\mathsf{Sub}(1)$,
\[
\tilde{\E}(B_q) = \E/q\,.
\]
This takes the place of localization.  Note that it also ``inverts'' all those elements $p\in \mathsf{Sub}(1)$ with $q\leq p$. 
\bigskip

For the global sections $\Gamma$, we have:
$$\Gamma(\tilde{\E}) \cong \tilde{\E}(B_\top) = \E/1 \cong \E\,.$$
So the topos of global sections of $\tilde{\E}$ is indeed  isomorphic to $\E$.
%%\end{frame}
%%%%%%%%%%%%%%%%%%%%%%%%%%%%%%%%%%%%%%%%%%%%%%%%%%%%%%%%%
%%%%%%%%%%%%%%%%%%%%%%%%%%%%%%%%%%%%%%%%%%%%%%%%%%%%%%%%%
%\section{Lambek-Moerdijk sheaf representation for toposes}

The \myemph{stalk} $\tilde{\E}_P$ at a prime ideal $P\in \mathsf{Spec}(\E)$ is the filter-quotient topos,
\[
\tilde{\E}_P = \varinjlim_{q\not\in P} \E/q, 
\]
at the prime \myemph{filter} $\mathsf{Sub}(1)\!\setminus\! P$.  
\medskip

One then has:
\[
\mathsf{Sub}_{\tilde{\E}_P}(1)\cong P\,,
\]
so the stalk topos $\tilde{\E}_P$ is indeed sublocal.

%%\end{frame}
%%%%%%%%%%%%%%%%%%%%%%%%%%%%%%%%%%%%%%%%%%%%%%%%%%%%%%%%%
%%%%%%%%%%%%%%%%%%%%%%%%%%%%%%%%%%%%%%%%%%%%%%%%%%%%%%%%%
%\section{Lambek-Moerdijk sheaf representation for toposes}

Again, there is always an injection from the global sections into the product of the stalks,
\[
\E \cong \Gamma(\tilde{\E}) \rightarrowtail \prod_{P\in X}\tilde{\E}_P\,.
\]

\begin{corollary}[Sub-direct-product representation for toposes]
Every topos $\E$ is isomorphic to a \myemph{sub}topos of a \myemph{direct product} of sublocal toposes.
\end{corollary}


%%\end{frame}
%%%%%%%%%%%%%%%%%%%%%%%%%%%%%%%%%%%%%%%%%%%%%%%%%%%%%%%%%
%%%%%%%%%%%%%%%%%%%%%%%%%%%%%%%%%%%%%%%%%%%%%%%%%%%%%%%%%
%\section{Lambek-Moerdijk sheaf representation for toposes}

We have the following \myemph{logical interpretation} of the sheaf representation:
%
\begin{itemize}
\item A topos $\E$ is (the term model of) a theory in Intuitionistic Higher-Order Logic.
\item A sublocal topos $\mathcal{S}$ is one that has the \emph{disjunction property}:
\[
\mathcal{S}\vdash p\vee q \qquad\text{iff}\qquad \mathcal{S}\vdash p\  \ \text{or}\  \ \mathcal{S}\vdash q\,,
\]
for all ``propositions'' $p, q$.
\item  The subdirect-product embedding is a logical completeness theorem with respect to such ``semantic'' toposes $\mathcal{S}$.
\item The sheaf representation is a Kripke-style completeness theorem for IHOL, with $\tilde\E$ as a ``sheaf of possible worlds''.
\end{itemize}

%%\end{frame}
%%%%%%%%%%%%%%%%%%%%%%%%%%%%%%%%%%%%%%%%%%%%%%%%%%%%%%%%%
%%%%%%%%%%%%%%%%%%%%%%%%%%%%%%%%%%%%%%%%%%%%%%%%%%%%%%%%%
\subsection{Lambek's modified sheaf representation for toposes}

But this result is \myemph{not entirely satisfactory}, because we would like the ``semantic worlds'' $\mathcal{S}$ to also have the \emph{existence property}:
\[
\mathcal{S}\vdash (\exists x:A)\varphi(x) \qquad\text{iff}\qquad  \mathcal{S}\vdash \varphi(a)\ \text{for some closed $a:A$}\,,
\] 
(we know that we can prove completeness with respect to such).
\medskip

\begin{definition}
A topos $\mathcal{S}$ is called \myemph{local} if the terminal object $1$ is indecomposable and projective, i.e.\ the global sections functor 
\[
\Gamma = \hom_\mathcal{S}(1, - ) : \mathcal{S} \to \Set
\]
preserves coproducts and epimorphisms.
\end{definition}
\medskip

Note that a local topos has \myemph{both} the disjunction and existence properties.

%%\end{frame}
%%%%%%%%%%%%%%%%%%%%%%%%%%%%%%%%%%%%%%%%%%%%%%%%%%%%%%%%%
%%%%%%%%%%%%%%%%%%%%%%%%%%%%%%%%%%%%%%%%%%%%%%%%%%%%%%%%%
%\section{Lambek's modified sheaf representation for toposes}

Lambek gave the following improvement over the sublocal sheaf representation:

\begin{theorem}[Lambek 1989]
Let $\E$ be a topos.  
There is a faithful logical functor $\E\rightarrowtail\mathcal{F}$ 
and a space $X$ with a sheaf of toposes 
$\tilde{\mathcal{F}}$ such that:
\begin{enumerate}
\item for every $p\in X$, the stalk $\tilde{\mathcal{F}}_p$ is a \myemph{local} topos, 
\item for the topos of global sections, we have: $\Gamma(\tilde{\mathcal{F}}) \cong \mathcal{F}$.
\end{enumerate}
Thus every topos is a \myemph{subtopos} of one that is isomorphic to the topos of global sections of a sheaf of \myemph{local} toposes.  
\end{theorem}
\medskip

This suffices for a \emph{sub-direct-product representation} into \myemph{local} toposes, and therefore gives the desired \emph{logical completeness} with respect to \myemph{local} toposes.  

But conceptually it is still not entirely satisfactory.

%%\end{frame}
%%%%%%%%%%%%%%%%%%%%%%%%%%%%%%%%%%%%%%%%%%%%%%%%%%%%%%%%%
%%%%%%%%%%%%%%%%%%%%%%%%%%%%%%%%%%%%%%%%%%%%%%%%%%%%%%%%%
\section{Local sheaf representation for toposes}

In my thesis, I proved:

\begin{theorem}[A. 1998]
Let $\E$ be a topos.  
There is a space $X$ with a sheaf of toposes $\tilde{\E}$ such that:
\begin{enumerate}
\item for every $p\in X$, the stalk $\tilde{\E}_p$ is a \myemph{local} topos, 
\item for the topos of global sections, we have: $\Gamma(\tilde{\E}) \cong \E$.
\end{enumerate}
Thus every topos is isomorphic to the global sections of a sheaf of \myemph{local} toposes.  
\end{theorem}
\medskip

As before, this gives a \emph{sub-direct-product representation},
\[
\E \rightarrowtail \prod_{p}\mathcal{S}_p
\]
into a product of local toposes $\mathcal{S}_p$, and therefore implies the desired \emph{logical completeness} of IHOL with respect to local toposes.  
%%\end{frame}
%%%%%%%%%%%%%%%%%%%%%%%%%%%%%%%%%%%%%%%%%%%%%%%%%%%%%%%%%
%%%%%%%%%%%%%%%%%%%%%%%%%%%%%%%%%%%%%%%%%%%%%%%%%%%%%%%%%
%\section{Local sheaf representation for toposes}

The stronger result also gives better ``Kripke semantics'' for IHOL, since the ``sheaf of possible worlds'' now has \myemph{local} stalks.
\medskip

For \myemph{classical} higher-order logic, more can be said:

\begin{lemma}
Every local \myemph{boolean} topos is well-pointed, i.e.\ the global sections functor,
\[
\Gamma = \hom_\mathcal{S}(1, - ) : \mathcal{S} \to \Set
\]
is faithful.
\end{lemma}

A well-pointed topos is essentially a model of set theory.  

\begin{corollary}
Every boolean topos is isomorphic to the global sections of a sheaf of \myemph{well-pointed} toposes.  
\end{corollary}

%%\end{frame}
%%%%%%%%%%%%%%%%%%%%%%%%%%%%%%%%%%%%%%%%%%%%%%%%%%%%%%%%%
%%%%%%%%%%%%%%%%%%%%%%%%%%%%%%%%%%%%%%%%%%%%%%%%%%%%%%%%%
%\section{Local sheaf representation for toposes}

For boolean toposes, we therefore have the  representation, 
\[
\mathcal{B} \rightarrowtail \prod_{p}\mathcal{S}_p 
\]
as sub-direct-product of \emph{well-pointed} toposes $\mathcal{S}_p$, along with its logical counterpart:
%
\begin{corollary}
Classical HOL is complete with respect to models in well-pointed toposes.
\end{corollary}
\medskip

These are \myemph{standard} models of classical HOL, taken in varying (``non-standard'') models of set theory.

%\end{frame}
%%%%%%%%%%%%%%%%%%%%%%%%%%%%%%%%%%%%%%%%%%%%%%%%%%%%%%%%%
%%%%%%%%%%%%%%%%%%%%%%%%%%%%%%%%%%%%%%%%%%%%%%%%%%%%%%%%%
%\section{Local sheaf representation for toposes}


Taking the global sections $\Gamma:\mathcal{S}_p \rightarrowtail \Set$ of each such well-pointed model then embeds any boolean topos $\mathcal{B}$ into a power of $\Set$:
\[
\mathcal{B} \rightarrowtail \prod_{p}\mathcal{S}_p \rightarrowtail \prod_{p}\Set_p \cong \Set^X\,,
\]
The various composites $\mathcal{B} \rightarrow \mathcal{S}_p \rightarrowtail \Set$ are Henkin style, ``non-standard'' models of HOL in $\Set$.

\begin{corollary}
Classical HOL is complete with respect to Henkin models in $\Set$.
\end{corollary}
\medskip

These Henkin models can be taken as the points of the space $X_\E$ for the sheaf representation.

%\end{frame}
%%%%%%%%%%%%%%%%%%%%%%%%%%%%%%%%%%%%%%%%%%%%%%%%%%%%%%%%%
%%%%%%%%%%%%%%%%%%%%%%%%%%%%%%%%%%%%%%%%%%%%%%%%%%%%%%%%%
%\section{Local sheaf representation for toposes}
To define the \myemph{space $X_\E$ of models}:
\medskip

In the \myemph{sublocal} case, the points were \emph{prime ideals} $p\subseteq\mathsf{Sub}(1)$.
These correspond exactly to \emph{lattice homomorphisms} $$p: \mathsf{Sub}_{\E}(1)\to \mathbf{2}\,.$$

For the \myemph{local} case, we instead take \emph{coherent functors} $$P: \E\to\Set\,.$$
These correspond exactly to Henkin models of (the theory represented by) $\E$.
\medskip

The \myemph{topology} is given (roughly speaking) by basic open sets of the following form, for all formulas $\varphi$:
\[
V_\varphi = \{ P\ |\ P\models \varphi \}
\]

%\end{frame}
%%%%%%%%%%%%%%%%%%%%%%%%%%%%%%%%%%%%%%%%%%%%%%%%%%%%%%%%%
%%%%%%%%%%%%%%%%%%%%%%%%%%%%%%%%%%%%%%%%%%%%%%%%%%%%%%%%%
%\section{Local sheaf representation for toposes}

The \myemph{structure sheaf} $\tilde\E$ is  first defined as a \myemph{stack} on $\E$ by ``slicing'',
\[
\tilde\E(A)\ =\ \E/A\,.
\] 
The stack is first strictified to a \myemph{sheaf}, and then transferred from $\E$ to the space $X_\E$ of models using a topos-theoretic covering theorem due to Butz and Moerdijk.
\medskip

For the \myemph{global sections} $\Gamma$, we then have:
$$\Gamma(\tilde{\E})\ \simeq\ \E/1\ \cong\ \E\,.$$

And for the \myemph{stalks} $\tilde{\E}_P$ we have the colimit,
\[
\tilde{\E}_P\ =\ \varinjlim_{A\,\in\int\!{P}} \E/A, 
\]
where the (filtered!) category of elements $\int\!{P}$ of the Henkin model $P$ takes the place of the prime filter.  






\end{document}