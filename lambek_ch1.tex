\documentclass[lambek.tex]{subfiles}


\begin{document}

\section*{Preface}

Shortly after finishing my PhD thesis, I received a friendly letter from Professor Lambek, in which he expressed interest in a result of mine that extended his work with Moerdijk \cite{LM1982}. He later cited my result in some papers on the philosophy of mathematics (including \cite{L1989,L2004}), in which he developed a congenial  position that attempted to reconcile the different conventional ones, on the basis of results concerning the free topos, the sheaf representations considered here, and related considerations from categorical logic.    

The particular result in question, discussed in section \ref{section:localtopos} below, extends prior results by Lambek and Moerdijk \cite{LM} and Lambek \cite{L1989}, and was later extended further in joint work with my  PhD students, first Henrik Forssell, and then Spencer Breiner.  This line of thought is, however, connected to a deeper one in modern mathematics, as I originally learned from the papers of Lambek.  That insight inspired  my  original contribution and also the later joint work with my students, and it continues to fascinate and inspire me.   The purpose of this survey is to present that line of thought, which owes more to Lambek than to anyone else.  

The main idea, in a nutshell, is that the ground-breaking duality theories developed in the mid-20th century can also be applied to logic, via its algebraization under categorical logic, and they thereby result in known and new completeness theorems.  This insight, which is already quite remarkable, can as it turns out be taken even further --- via what is now known as ``categorification'' --- to establish an even deeper relation between logic and geometry, a glimpse of which can also be had in topos theory, and elsewhere.

\section{Gelfand duality}

Perhaps the ur-example of the sort of duality theory we have in mind is that between topological spaces and commutative rings given by Gelfand duality.  To give a brief (and ahistorical) sketch, let $X$ be a  space and consider the ring of real-valued functions on $X$, with pointwise algebraic operations,
\[
 \mathcal{C}(X)\ =\ \mathsf{Top}(X, \mathbb{R}).
 \]
 This construction is a (contravariant) functor from ``geometry'' to ``algebra'',
 \[
 \mathcal{C} : \mathsf{Top}^\mathsf{op} \to \mathsf{CRng}.
 \]
%
The functor $ \mathcal{C}$ can be shown (see \cite{johnstone}, Ch.~4) to be full and faithful if we restrict to compact Hausdorff spaces $X$ and (necessarily bounded) continuous functions  $\mathcal{C}^*(X)$,
 \[
 \mathcal{C}^* : \mathsf{KHaus}^\mathsf{op} \hookrightarrow \mathsf{CRng}.
 \]
It then requires some further work to determine exactly \emph{which} rings are of the form $\mathcal{C}^*(X)$ for some space $X$.   These are called \emph{$C^*$-algebras}, and they can be defined as commutative rings $A$ satisfying the following conditions (\cite{johnstone} 4.4):
\begin{enumerate}
\item the additive group of $A$ is divisible and torsion free,
\item $A$ has a partial order compatible with the ring structure and such that $a^2 \geq 0$ for all $a\in A$,
\item $A$ is Archemedian, i.e.\ for every  $a\in A$ there is an integer $n$ such that $n\cdot 1_A \geq a$,
\item if $1_A\geq na$ for all positive integers $n$, then $a\leq 0$,
\item $A$ is complete in the norm $$||a|| = \inf\{ q\in \mathbb{Q}^+\ |\ q\cdot 1_A \geq a\text{ and } q\cdot 1_A \geq -a\}\,.$$
\end{enumerate}
There are many equivalent specifications (most using complex numbers in place of reals).

 \begin{theorem*}[Gelfand duality]
 The category $\mathsf{KHaus}$ of compact Hausdorff spaces is dual to the category $\mathsf{C^*Alg}$ of $C^*$-algebras and their homomorphisms, via the functor $\mathcal{C}^*$:
 \[
 \mathsf{KHaus}^\mathsf{op} \simeq \mathsf{C^*Alg}.
 \]
 \end{theorem*}

How can we  recover the space $X$ from its ring of functions $\mathcal{C}^*(X)$?

\begin{itemize}
\item The points  $x\in X$ determine maximal ideals in the ring $\mathcal{C}^*(X)$,
\[
M_x = \{\, f : X\to \mathbb{R}\ |\ f(x) = 0\, \}\,,
\]
and every maximal ideal in $\mathcal{C}^*(X)$ is of this form for a unique $x\in X$.

\item For any ring $A$, the (Zariski) topology on the set $\mathsf{Max}(A)$ of maximal ideals has a basis of open sets of the form:
\[
B_a\, =\, \{M \in X\ |\ a\notin M\, \}\, ,\qquad a\in A.
\]

\item If $A$ is a \emph{$C^*$-algebra}, then this specification  determines a compact Hausdorf space $X = \mathsf{Max}(A)$ such that $A\cong \mathcal{C}^*(X)$.
\end{itemize}

A key step in the proof is the following:
%
 \begin{theorem*}
 Let $A$ be a $C^*$-algebra.  For any maximal ideal $M$ in $A$, the quotient field 
 $A/M$ is isomorphic to  $\mathbb{R}$. 
    \end{theorem*}
%   
It follows that there is an injection of rings,
$$A \mono \prod_{M\in \mathsf{Max}(A)}\!\!A/M\ \cong\ \mathbb{R}^{\mathsf{Max}(A)}\,,$$
the image of which  can be shown to consist of the Zariski continuous functions, i.e.\  $\mathcal{C}^*(\mathsf{Max}(A))$.


% \begin{theorem*}[Gelfand-Stone-Naimark]
% $\mathsf{KHaus}$ is dual to the category of $C^*$-algebras,
% \[
%\mathsf{KHaus}^\mathsf{op}\ \simeq\ C^*\mathsf{Alg}.
% \]
% \end{theorem*}

%%\end{frame}
%%%%%%%%%%%%%%%%%%%%%%%%%%%%%%%%%%%%%%%%%%%%%%%%%%%%%%%%%
%%%%%%%%%%%%%%%%%%%%%%%%%%%%%%%%%%%%%%%%%%%%%%%%%%%%%%%%%
\section{Grothendieck's sheaf representation for commutative rings}

Grothendieck extended the Gelfand duality from $C^*$-algebras to \emph{all} commutative rings by generalizing on the ``geometric'' side from compact Hausdorff spaces to the new notion of \emph{(affine) schemes},
 \[
\mathsf{Scheme}_\mathsf{aff}^\mathsf{op}\ \simeq\ \mathsf{CRng}.
 \]
The essential difference is to generalize the ``ring of values'' from the constant ring $\mathbb{R}$ to a ring $\mathcal{R}$ that ``varies continuously over the space $X$'', i.e.\ a \emph{sheaf of rings}.   
The various rings $\mathcal{R}_x$ that are the stalks of $\mathcal{R}$ generalize the \emph{local rings} of real-valued functions that vanish at the points $x\in X$.
This change allows \emph{every} commutative ring $A$ to be seen as a ring of continuous functions on a suitable space $X_A$ (the prime spectrum of $A$), where the values of the functions are in a suitable sheaf of (local) rings $\mathcal{R}$ on $X_A$.

%%\end{frame}
%%%%%%%%%%%%%%%%%%%%%%%%%%%%%%%%%%%%%%%%%%%%%%%%%%%%%%%%%
%%%%%%%%%%%%%%%%%%%%%%%%%%%%%%%%%%%%%%%%%%%%%%%%%%%%%%%%%
%\section{Grothendieck's sheaf representation for commutative rings}

\begin{definition} A  ring is called \emph{local} if it has a unique maximal ideal. \\
Equivalently: 
\begin{equation}\label{eq:localring}
\text{$x+y$ is a unit}\quad\text{implies}\quad\text{$x$ is a unit or $y$ is a unit}.
\end{equation}
\end{definition}
%
\begin{theorem}[Grothendieck]
Let $A$ be a ring.  There is a space $X$ with a sheaf of rings $\mathcal{R}$ such that:
\begin{enumerate}
\item for every $p\in X$, the stalk $\mathcal{R}_p$ is a local ring, 
\item there an isomorphism, $$A\cong \Gamma(\mathcal{R})\,,$$
where $\Gamma(\mathcal{R})$ is the ring of global sections.
\end{enumerate}
Thus every ring is isomorphic to the ring of global sections of a sheaf of local rings.
\end{theorem}

%%\end{frame}
%%%%%%%%%%%%%%%%%%%%%%%%%%%%%%%%%%%%%%%%%%%%%%%%%%%%%%%%%
%%%%%%%%%%%%%%%%%%%%%%%%%%%%%%%%%%%%%%%%%%%%%%%%%%%%%%%%%
%\section{Grothendieck's sheaf representation for commutative rings}

The \myemph{space} $X$ in the theorem is the \emph{prime spectrum} $\mathsf{Spec}(A)$ of $A$:
 \begin{itemize}
\item points $p\in \mathsf{Spec}(A)$ are prime ideals $p\subseteq A$,
\item the (Zariski) topology has basic opens of the form:
 $$B_f = \{ p\in \mathsf{Spec}(A)\ |\ f\not\in p \}, \quad f\in A.$$
\end{itemize}
Note the similarity to the space $\mathsf{Max}(A)$ of maximal ideals from the Gelfand case.  
Unlike that case, however, the functor 
\[
\mathsf{Spec} : \mathsf{CRng}^\mathsf{op} \to \mathsf{Top}
 \]
is not full, and so we need to equip the spaces  $\mathsf{Spec}(A)$ with an additional structure.

The \myemph{structure sheaf} $\mathcal {R}$ is determined by ``localizing'' $A$ at $f$,
\[
\mathcal{R}(B_f) = [f]^{-1}A
\]
where $A \rightarrow [f]^{-1}A$ freely inverts all of the elements $f, f^2, f^3, \dots$.
\medskip

The \myemph{stalk} $\mathcal {R}_p$ of this sheaf at a point $p\in\mathsf{Spec}(A)$ is then seen to be the localization of $A$ at $S = A\setminus p$,
\[
\mathcal{R}_p = S^{-1}A\,.
\]

%%\end{frame}
%%%%%%%%%%%%%%%%%%%%%%%%%%%%%%%%%%%%%%%%%%%%%%%%%%%%%%%%%
%%%%%%%%%%%%%%%%%%%%%%%%%%%%%%%%%%%%%%%%%%%%%%%%%%%%%%%%%
%\section{Grothendieck's sheaf representation for commutative rings}

The \myemph{affine scheme} $(\mathsf{Spec}(A), \mathcal {R})$ presents $A$ as a ``ring of continuous functions'' in the following sense: 
\begin{itemize}
\item each element $f\in A$ determines a ``continuous function'',
\[
\hat{f} : \mathsf{Spec}(A) \to \mathcal {R}\,,
\]
except that the ring $\mathcal{R}$ is itself ``varying continuously over the space $\mathsf{Spec}(A)$'' -- i.e.\ it is a sheaf -- and the function $\hat{f}$ is then a global section of the sheaf $\mathcal{R}$.  

\item Each stalk $\mathcal{R}_p$ is a local ring, and thus has a  unique maximal ideal, consisting of ``those functions  $\hat{f} : \mathsf{Spec}(A) \to \mathcal{R}$ that vanish at $p$''.

\item $(\mathsf{Spec}(A), \mathcal {R})$ is a ``representation'' of $A$ in the sense that $f\mapsto\hat{f}$ is an isomorphism of rings
\[
A \cong \Gamma(\mathcal{R})\,.
\]
\end{itemize}

Since there is always an injective homomorphism from the global sections of a sheaf into the product of all the stalks,
\[
\Gamma(\mathcal{R}) \rightarrowtail \prod_{p}\mathcal{R}_p \,,
\]
we have the following:

\begin{corollary}[``Subdirect-product representation'']
Every ring $A$ is isomorphic to a {sub}ring of a ``direct product'' of local rings.
I.e.\ there is an injective ring homomorphism
\[
A \rightarrowtail \prod_{p}\mathcal{R}_p \,,
\]
where the $\mathcal{R}_p$ are all local rings.
\end{corollary}

%%\end{frame}
%%%%%%%%%%%%%%%%%%%%%%%%%%%%%%%%%%%%%%%%%%%%%%%%%%%%%%%%%
%%%%%%%%%%%%%%%%%%%%%%%%%%%%%%%%%%%%%%%%%%%%%%%%%%%%%%%%%
\section{Lambek-Moerdijk sheaf representation for toposes}

\begin{definition} Call a  (small, elementary)  topos $\E$  \emph{sublocal}\footnote{
In the original work \cite{LM}, and elsewhere, the term \emph{local} was used for the concept here called \emph{sublocal}, and another term was then required for the stronger conditional that we call \emph{local} in Definition \ref{def:localtopos} below.
} 
if its subterminal lattice $\mathsf{Sub}_\E(1)$ has a unique maximal ideal.
Equivalently, for $x,y\in \mathsf{Sub}_\E(1)$: 
\[
x\vee y = 1\quad\text{implies}\quad x=1\ \text{or}\ y=1\,.
\]
\end{definition}
%
Note the formal analogy to the concept of local ring.
In \cite{LM} the following analogue of the Grothendieck sheaf representatation for rings is given for toposes (henceforth, \emph{topos} unqualified will mean small, elementary topos):
\begin{theorem}
Let $\E$ be a topos.  There is a space $X$ with a sheaf of toposes $\tilde\E$ such that:
\begin{enumerate}
\item for every $p\in X$, the stalk $\tilde\E_p$ is a sublocal topos, 
\item for the topos $\Gamma(\tilde\E)$ of global sections, we have an isomorphism, $$\E\cong \Gamma(\tilde\E)\,.$$
\end{enumerate}
Thus every topos is isomorphic to the topos of global sections of a sheaf of sublocal toposes.
\end{theorem}

%%\end{frame}
%%%%%%%%%%%%%%%%%%%%%%%%%%%%%%%%%%%%%%%%%%%%%%%%%%%%%%%%%
%%%%%%%%%%%%%%%%%%%%%%%%%%%%%%%%%%%%%%%%%%%%%%%%%%%%%%%%%
%\section{Lambek-Moerdijk sheaf representation for toposes}

The space $X$ mentioned in the theorem is what may be called the \emph{subspectrum of the topos},  $X=\mathsf{sSpec}(\E)$;   
it is just the prime ideal spectrum of the distributive lattice $\mathsf{Sub}(1)$:
\begin{itemize}
\item the points $P\in \mathsf{sSpec}(\E)$ are prime ideals $P\subseteq \mathsf{Sub}(1)$,
\item the topology has basic opens of the form:
$$B_q = \{ P\in \mathsf{Spec}(\E)\ |\ q\not\in P \}, \quad q\in\mathsf{Sub}(1)\,.$$
\end{itemize}
%
Note the close analogy to the space $\mathsf{Spec}(A)$ for a commutative ring $A$.

The lattice of all open sets of $\mathsf{sSpec}(\E)$ is then (isomorphic to) the ideal completion of the lattice $\mathsf{Sub}(1)$,
$$\mathcal{O}(\mathsf{Spec}(\E)) = \mathsf{Idl}(\mathsf{Sub}(1))\,.$$

%Equivalently, we could have set:
%\begin{enumerate}
%\item points $F\in Spec(\E)$ are prime filters $F\subseteq \mathsf{Sub}(1)$,
%\item the topology has basic opens of the following form, for $q\in\mathsf{Sub}(1)$:
%$$D_q = \{ F\in Spec(\E)\ \ |\ q\in F \}.$$
%\end{enumerate}
%

%%\end{frame}
%%%%%%%%%%%%%%%%%%%%%%%%%%%%%%%%%%%%%%%%%%%%%%%%%%%%%%%%%
%%%%%%%%%%%%%%%%%%%%%%%%%%%%%%%%%%%%%%%%%%%%%%%%%%%%%%%%%
%\section{Lambek-Moerdijk sheaf representation for toposes}


Next, let us define a \emph{structure sheaf} $\tilde{\E}$ on $\mathsf{sSpec}(\E)$ by ``slicing'' $\E$ at $q \in\mathsf{Sub}(1)$,
\[
\tilde{\E}(B_q) = \E/q\,.
\]
This takes the place of the localization of a ring $A$ at a basic open $B_f$:
\[
\mathcal{R}_A(B_f) = [f]^{-1}A\,.
\]
Note that $\E/q$ also ``inverts'' all those elements $p\in \mathsf{Sub}(1)$ with $q\leq p$, in the sense that the canonical map 
$q^* : \E \to \E/q$ takes every $p \mono 1$ to $q\wedge p \mono q$, and so if $q\leq p$ then $q^*p = 1_q : q \mono q$.

The fact that $\tilde{\E}$ is indeed a sheaf on $\mathsf{sSpec}(\E)$ comes down to showing that, for any $p, q \in \mathsf{Sub}(1)$, there is a canonical equalizer of toposes (and logical morphisms),
\[
\E/p\vee q\ \mono\ \E/p \times \E/q\ \rightrightarrows\ \E/p\wedge q \,.
\]
This in turn says that in a diagram of the form: 
% \[ 
% \xymatrix{ C \ar[d]_{i} \ar[rr]^{c} && F \ar[d]^{f} \\ D \ar[d]_{j} \ar[rr]_(.6){1_D} \ar@{.>}[rru]^d && D \\ E \ar@{.>}[rru]_{e} \ar@{.>}[rruu]^(.3){k} &&.} 
% \] 
 \[ 
 \xymatrix{ 
& X \ar@{>->}[ld] \ar[dd] \ar@{>->}[rr] && Q \ar@{.>}[ld] \ar[dd] & \\
 P \ar[dd] \ar@{.>}[rr] && Y \ar@{.>}[dd]\\ 
 & p \wedge q \ar@{>->}[ld] \ar@{>->}[rr] && q \ar@{>->}[ld] & \\
  p \ar@{>->}[rr] && p\vee q .} 
 \] 
 with a pushout-pullback of monos in the base and the two vertical squares involving $X$ given as pullbacks, one can complete the cube as indicating by first forming the pushout $Y$ on the top and then obtaining the front vertical map from $Y$, and the resulting new vertical faces will then also be pullbacks. This is a rather special ``descent condition'' for the presheaf $\E/-$.
 
The stalk $\tilde{\E}_P$ of this sheaf at a point $P\in \mathsf{sSpec}(\E)$ is computed as the filter-quotient of $\E$ over the complement of the prime ideal $P\subseteq  \mathsf{Sub}_{\E}(1)$, i.e.\ the prime filter $P^c = \mathsf{Sub}(1)\!\setminus\! P$.  Thus for the stalk we have the (filtered) colimit (taken in $\mathsf{Cat}$, but again a topos):
\[
\tilde{\E}_P = \varinjlim_{q\not\in P} \E/q\,.
\]
For this stalk topos, one then has the subterminal lattice:
\[
\mathsf{Sub}_{\tilde{\E}_P}(1)\cong \mathsf{Sub}_{\E}(1)/P^c\,,
\]
where $\mathsf{Sub}_{\E}(1)/P^c$ is the quotient Heyting algebra by the prime filter $P^c$.  Since for the prime filter $P^c$ we have $p \vee q \in P^c$ implies $p \in P^c$ or $q \in P^c$, it thus follows that the stalk topos $\tilde{\E}_P$ is indeed sublocal.


Finally, for the global sections of $\tilde{\E}$ we  have simply:
$$\Gamma(\tilde{\E}) \cong \tilde{\E}(B_\top) = \E/1 \cong \E\,.$$
Thus the topos of global sections of $\tilde{\E}$ is indeed isomorphic to the original topos $\E$.  In this way, $\E$ is isomorphic to the global sections of a sheaf of sublocal toposes.
\medskip

Again, there is always an injection from the global sections into the product of the stalks, which in this case gives a conservative logical morphism of the form
\[
\E \cong \Gamma(\tilde{\E}) \rightarrowtail \prod_{P\in \mathsf{sSpec}(\E)}\!\tilde{\E}_P\,.
\]

\begin{corollary}\label{cor:SDPtopos}
Every topos has a conservative logical morphism into a product of sublocal toposes.
\end{corollary}

%%\end{frame}
%%%%%%%%%%%%%%%%%%%%%%%%%%%%%%%%%%%%%%%%%%%%%%%%%%%%%%%%%
%%%%%%%%%%%%%%%%%%%%%%%%%%%%%%%%%%%%%%%%%%%%%%%%%%%%%%%%%
\subsection{Lambek's modified sheaf representation for toposes}


We have the following \emph{logical interpretation} of the sheaf representation theorem for toposes and its corollary.
%
\begin{itemize}
\item A topos $\E$ can be regarded as the syntactic category $\E_\mathbb{T}$ of a theory $\mathbb{T}$ in Intuitionistic Higher-Order Logic (IHOL).  Thus for any sentence $\phi$ in the language of the theory $\mathbb{T}$,
\[ 
\E_\mathbb{T}\models \phi \qquad\text{iff}\qquad \mathbb{T} \vdash \phi\,.
\]

\item A sublocal topos $\mathcal{S}$ is one that has the \emph{disjunction property}:
\[
\mathcal{S}\models \phi\vee \psi \qquad\text{iff}\qquad \mathcal{S}\models \phi\  \ \text{or}\  \ \mathcal{S}\models \psi\,,
\]
for all sentences $\phi, \psi$.

\item  The ``subdirect-product theorem'' given by Corollary \ref{cor:SDPtopos} is a logical completeness theorem with respect to interpretations $\E_\mathbb{T}\rightarrow\mathcal{S}$ of $\mathbb{T}$ into sublocal toposes $\mathcal{S}$.  It says that, for any theory $\mathbb{T}$ in IHOL, a sentence $\phi$ is provable, $\mathbb{T}\vdash \phi$,  iff it is true in every interpretation of $\mathbb{T}$ in a sublocal topos $\mathcal{S}$.  Thus IHOL is complete with respect to models in sublocal toposes.

\item The sheaf representation is a Kripke-style completeness theorem for IHOL, with $\tilde\E$ as a ``sheaf of possible worlds'' (see \cite{Lambeck:sheaf of possible worlds}).
\end{itemize}

Under this interpretation, however, the present sheaf representation is not entirely satisfactory, because we would really like the ``semantic toposes'' $\mathcal{S}$ to be even more $\Set$-like, in addition to being sublocal, by also having the \emph{existence property}:
\[
\mathcal{S}\models (\exists x:A)\varphi(x) \qquad\text{iff}\qquad  \mathcal{S}\models \varphi(a)\text{, for some closed $a:A$}\,.
\] 

\begin{definition}\label{def:localtopos}
A topos $\mathcal{S}$ will be called \emph{local} if the terminal object $1$ is both indecomposable and projective, i.e.\ the global sections functor 
\[
\Gamma = \hom_\mathcal{S}(1, - ) : \mathcal{S} \to \Set
\]
preserves coproducts and epimorphisms. Note that a local topos is exactly one that has both the disjunction and existence properties.
\end{definition}

In the paper \cite{Lambeck1988}, Lambek gave the following improvement over the sublocal sheaf representation:

\begin{theorem}[Lambek 1989]
Let $\E$ be a topos.  
There is a faithful logical functor $\E\rightarrowtail\mathcal{F}$ 
and a space $X$ with a sheaf of toposes 
$\tilde{\mathcal{F}}$ such that:
\begin{enumerate}
\item for every $p\in X$, the stalk $\tilde{\mathcal{F}}_p$ is a  local topos, 
\item for the topos of global sections, we have and isomorphism $\mathcal{F} \cong \Gamma(\tilde{\mathcal{F}}) $.
\end{enumerate}
Thus every topos is a  subtopos of one that is isomorphic to the global sections of a sheaf of  local toposes.  
\end{theorem}
\medskip
%
The proof was inspired by the Henkin completeness theorem for higher-order logic, and first performs a sort of ``Henkinization'' of $\E$ to get bigger topos $\mathcal{F}$ with witnesses for all existential quantifiers, in a suitable sense. This result then suffices for a subdirect-product embedding of any topos $\E$ into a product of \emph{local} toposes, and therefore gives the desired logical completeness of IHOL with respect to such toposes, which are much more $\Set$-like.

%%%%%%%%%%%%%%%%%%%%%%%%%%%%%%%%%%%%%%%%%%%%%%%%%%%%%%%%%
\section{Local sheaf representation for toposes}\label{section:localtopos}

The main result of my PhD thesis \cite{A1998} was this:
\begin{theorem}\label{theorem:local}
Let $\E$ be a topos.  
There is a space $X$ with a sheaf of toposes $\tilde{\E}$ such that:
\begin{enumerate}
\item for every $p\in X$, the stalk $\tilde{\E}_p$ is a local topos, 
\item for the topos of global sections,  there is an equivalence $\E\simeq \Gamma(\tilde{\E})$.
\end{enumerate}
Thus every topos is equivalent to the global sections of a sheaf of local toposes.  
\end{theorem}

As before, this gives a \emph{subdirect-product representation},
\[
\E \rightarrowtail \prod_{p\,\in X}\mathcal{S}_p
\]
into a product of local toposes $\mathcal{S}_p$, and therefore implies the desired logical completeness of IHOL with respect to local toposes.  
This stronger result also gives better ``Kripke semantics'' for IHOL, since the ``sheaf of possible worlds'' now has local stalks.
\medskip

For classical higher-order logic, something more can be said:

\begin{lemma}
Every boolean, local topos $\mathcal{S}$ is well-pointed, i.e.\ the global sections functor,
\[
\Gamma = \hom_\mathcal{S}(1, - ) : \mathcal{S} \to \Set
\]
is faithful.
\end{lemma}

\begin{corollary}
Every boolean topos is isomorphic to the global sections of a sheaf of well-pointed toposes.  
\end{corollary}

For boolean toposes $\mathcal{B}$, we therefore have an embedding, 
\[
\mathcal{B} \rightarrowtail \prod_{p\,\in X}\mathcal{S}_p 
\]
as subdirect-product of \emph{well-pointed} toposes $\mathcal{S}_p$.  The logical counterpart now says:
%
\begin{corollary}
Classical HOL is complete with respect to models in well-pointed toposes.
\end{corollary}
\medskip
A well-pointed topos is essentially a model of set theory.  Indeed, it is worth emphasizing that the models of HOL here are \emph{standard} models of classical HOL, taken in varying models $\mathcal{S}$ of set theory.

Finally, taking the global sections $\Gamma:\mathcal{S}_p \rightarrowtail \Set$ of each such well-pointed topos, we get a faithful functor from any boolean topos $\mathcal{B}$ into a power of $\Set$:
\[
\mathcal{B} \rightarrowtail \prod_{p\,\in X}\mathcal{S}_p \rightarrowtail \prod_{p\,\in X}\Set \cong \Set^X\,.
\]
However, the various composites $\mathcal{B} \rightarrow \mathcal{S}_p \rightarrowtail \Set$ are now not logical functors, because they need not preserve exponentials; they do, however, preserve the first-order logical structure (they are also exact; cf.~\cite{Freyd} theorem x.x).  Thus they are what the logician calls a ``Henkin'' or ``non-standard'' model of HOL in $\Set$.  In this way, we recover the familiar ``Henkin completeness theorem for HOL'' \cite{Henkin}:

\begin{corollary}
Classical HOL is complete with respect to Henkin models.
\end{corollary}

For the proof of the sheaf representation theorem \ref{theorem:local}, these ``Henkin models'' will be taken as the points of the space $X_\E$, which we call the \emph{space of models} (following \cite{ButzMoerdijk}).
In the  sublocal case, the points were the \emph{prime ideals} $P\subseteq\mathsf{Sub}(1)$.
These correspond exactly to the \emph{lattice homomorphisms} $$p: \mathsf{Sub}_{\E}(1)\to \mathbf{2}\,.$$

For the local case, we will instead take \emph{coherent functors} $$P: \E\to\Set\,.$$
These correspond exactly to (Henkin) models of the ``theory'' $\E$.\footnote{
The astute reader will note that there is an issue of size here: the collection of all such functors may be too big to form a \emph{set}.  The technical remedy is explained in the paper \cite{Awodey1998}.}

The topology on $X_\E$ is given (roughly speaking) by basic open sets of the following form, for all formulas $\varphi$:
\[
V_\varphi = \{ P\ |\ P\models \varphi \}
\]


The \myemph{structure sheaf} $\tilde\E$ is first defined as a \myemph{stack} on $\E$ by ``slicing'',
\[
\tilde\E(A)\ =\ \E/A\,.
\] 
The stack is first strictified to a \myemph{sheaf}, and then transferred from $\E$ to the space $X_\E$ of models using a topos-theoretic covering theorem due to Butz and Moerdijk.
\medskip

For the \myemph{global sections} $\Gamma$, we then have:
$$\Gamma(\tilde{\E})\ \simeq\ \E/1\ \cong\ \E\,.$$

And for the \myemph{stalks} $\tilde{\E}_P$ we have the colimit,
\[
\tilde{\E}_P\ =\ \varinjlim_{A\,\in\int\!{P}} \E/A, 
\]
where the (filtered!) category of elements $\int\!{P}$ of the Henkin model $P$ takes the place of the prime filter.  



\end{document}