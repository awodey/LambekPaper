

\documentclass[lambek.tex]{subfiles}




\begin{document}




\section{Sheaf representation for pretoposes (A.-Breiner)}

As for rings and affine schemes, we can equip the spectrum $\mathbf{Spec}(\mathcal{B})$ of the pretopos $\mathcal{B}$ with a ``structure sheaf'' $\tilde{\mathcal{B}}$,\\
  defined as in the sheaf representation for toposes: 
\begin{itemize}
\item start with the presheaf of categories $\tilde{\mathcal{B}} : \mathcal{B}^{\mathrm{op}}\to\mathbf{Cat}$ with,
\[
\tilde{\mathcal{B}}(X) \cong \mathcal{B}/X\,,
\]
for all $X\in \B$. This is a \myemph{stack} because $\mathcal{B}$ is a pretopos.
 
\item Strictify to get a sheaf of categories on $\mathcal{B}$. 

\item Use the equivalence of toposes,
\[
\mathbf{Sh}(\B)\ \simeq\ \mathbf{Sh}_{\mathsf{eq}}(\mathbf{Spec}(\mathcal{B}))
\]
$\mathbf{Spec}(\mathcal{B})$ is determined so that this equivalence holds.

\item Move $\tilde{\mathcal{B}}$ along this equivalence in order to get an equivariant sheaf on $\mathbf{Spec}(\mathcal{B})$.
\end{itemize}

%\end{frame}
%%%%%%%%%%%%%%%%%%%%%%%%%%%%%%%%%%%%%%%%%%%%%%%%%%%%%%%%%
%%%%%%%%%%%%%%%%%%%%%%%%%%%%%%%%%%%%%%%%%%%%%%%%%%%%%%%%%
%\section{Sheaf representation for pretoposes (A.-Breiner)}

The transported $\tilde{\mathcal{B}}$  is an equivariant sheaf of local, boolean pretoposes on the topological groupoid $\mathbf{Spec}(\mathcal{B})$.

\begin{itemize}
\item Logically, $\tilde{\mathcal{B}}$ is a sheaf of ``local theories'' on the groupoid of models, equipped with the logical topology.  

\item As before, $\tilde{\mathcal{B}}$ has global sections $\Gamma\tilde{\mathcal{B}} \simeq \mathcal{B}$.  So the original ``theory'' $\mathcal{B}$ is the ``theory of all models''.

\item The stalk $\tilde{\mathcal{B}}_P$ at a model $P : \mathcal{B} \to \Set$ is a well-pointed pretopos: it is the ``elementary diagram'' of the model $P$.
\medskip

\item The global sections functors $\Gamma_P : \tilde{\mathcal{B}}_P \rightarrowtail \Set$ are faithful \myemph{pretopos} morphisms, i.e.\ ``complete models''.

\end{itemize}
%\end{frame}
%%%%%%%%%%%%%%%%%%%%%%%%%%%%%%%%%%%%%%%%%%%%%%%%%%%%%%%%%
%%%%%%%%%%%%%%%%%%%%%%%%%%%%%%%%%%%%%%%%%%%%%%%%%%%%%%%%%
%\section{Sheaf representation for pretoposes (A.-Breiner)}

In sum, we have the following:

\begin{theorem}[A.-Breiner 2013]
Let $\B$ be a boolean pretopos.  
There is a topological groupoid $G$ with an equivariant sheaf of pretoposes $\tilde{\B}$ such that:
\begin{enumerate}
\item for every $x\in G$, the stalk $\tilde{\B}_x$ is a well-pointed pretopos, 
\item for the pretopos of global sections, we have: $\Gamma(\tilde{\B}) \cong \B$.
\end{enumerate}
Thus every Boolean pretopos is isomorphic to the global sections of a sheaf of well-pointed pretoposes.  
\end{theorem}
\medskip

There is an analogous result for the non-Boolean case, with local pretoposes in place of well-pointed ones in the stalks.

%\end{frame}
%%%%%%%%%%%%%%%%%%%%%%%%%%%%%%%%%%%%%%%%%%%%%%%%%%%%%%%%%
%%%%%%%%%%%%%%%%%%%%%%%%%%%%%%%%%%%%%%%%%%%%%%%%%%%%%%%%%
%\section{Sheaf representation for pretoposes (A.-Breiner)}

The resulting sub-direct-product representation $\mathcal{B}\rightarrowtail \prod_{x}\tilde{\mathcal{B}}_x$ yields:\\[1ex]

\begin{corollary}[G\"odel completeness theorem] 
There is a pretopos embedding,
\[
\mathcal{B}\rightarrowtail \prod_{x\in G}\tilde{\mathcal{B}}_{x} \rightarrowtail \prod_{x\in G}\Set \simeq \Set^{|G|}\,,
\]
with $|G|$ the set of points of the topological groupoid $G = \mathbf{Spec}(\mathcal{B})$ of models.
\end{corollary}
\medskip

%Other consequences of the sheaf representation include:
%
%\begin{itemize}
%\item Conceptual completeness (Makkai-Reyes)
%\item Beth definability (Makkai)
%\end{itemize}
%

%\end{frame}
%%%%%%%%%%%%%%%%%%%%%%%%%%%%%%%%%%%%%%%%%%%%%%%%%%%%%%%%%
%%%%%%%%%%%%%%%%%%%%%%%%%%%%%%%%%%%%%%%%%%%%%%%%%%%%%%%%%
\section{Logical schemes}

For a Boolean pretopos $\B$, call the pair $$(\mathbf{Spec}(\mathcal{B}), \tilde{\mathcal{B}})$$ an affine \myemph{logical scheme}.  \\

A morphism of logical schemes
\[
(h, \tilde{h}) : (\mathbf{Spec}(\mathcal{B}), \tilde{\mathcal{B}}) \to (\mathbf{Spec}(\mathcal{A}), \tilde{\mathcal{A}})
\]
consists of a continuous groupoid homomorphism 
\[
h : \mathbf{Spec}(\mathcal{B}) \to \mathbf{Spec}(\mathcal{A}),
\]
together with a pretopos functor 
\[
\tilde{h} : \tilde{\mathcal{A}} \to h_*\tilde{\mathcal{B}}
\]
over $\mathbf{Spec}(\mathcal{A})$.

%\end{frame}
%%%%%%%%%%%%%%%%%%%%%%%%%%%%%%%%%%%%%%%%%%%%%%%%%%%%%%%%%
%%%%%%%%%%%%%%%%%%%%%%%%%%%%%%%%%%%%%%%%%%%%%%%%%%%%%%%%%
%\section{Logical schemes}

\begin{theorem}[A.-Breiner 2012]
Every pretopos functor $\mathcal{A} \to \mathcal{B}$ induces a morphism of the associated affine logical schemes.  Moreover, the  functor
\[
\mathsf{Spec} : \mathbf{BPreTop}^{\mathsf{op}} \longrightarrow \mathbf{LScheme}_{aff}
\]
is \myemph{full}: every map of schemes comes from a map of pretoposes.
\end{theorem}

\begin{corollary}[First-order logical duality]
There is an equivalence,
\[
\mathbf{BPreTop}^{\mathsf{op}} \ \simeq\ \mathbf{LScheme}_{aff}\,.
\]
\end{corollary}

Thus the category of Boolean pretoposes is dual to the category of affine logical schemes.

\end{document}