%%%%%%%%%%%%%%%%%%%%%%%%%%%%%%%%%%%%%%%%%%%%%%
% Logical Schemes
% Steve Awodey
% August 2015
%%%%%%%%%%%%%%%%%%%%%%%%%%%%%%%%%%%%%%%%%%%%%

% AMS-Latex %
\documentclass{beamer}
%\documentclass[handout]{beamer}
\usetheme{default}
\usepackage{etex} 
\usepackage{amsmath,amssymb}
\usepackage{bussproofs}
\usepackage{listings}
\usepackage{relsize}
%\usepackage[nohug]{diagrams}
\usepackage[all]{xy}
\xyoption{2cell}
\xyoption{curve}
\UseTwocells
\input{diagxy}
\CompileMatrices      
\usepackage{tikz}
\usepackage{pdfpages}
 

\newcommand{\toarrow}{\ensuremath{\rightarrow}} 

\newcommand{\T}{\ensuremath{\mathbb{T}}}
\newcommand{\E}{\ensuremath{\mathcal{E}}} 
\newcommand{\B}{\ensuremath{\mathcal{B}}} 
\newcommand{\A}{A_\bullet}
\newcommand{\Id}{\mathrm{Id}}
\newcommand{\myemph}[1]{\textbf{#1}}    % Produces boldface text
\newcommand{\imp}{\ensuremath{\Rightarrow}} 
\newcommand{\arr}{\ensuremath{\rightarrow}} 


%%%
%%%  TYPE THEORETIC COMMANDS
%%%

\newcommand{\U}{\mathcal{U}}      
\newcommand{\Sn}{\mathbb{S}}       
\newcommand{\Z}{\mathbb{Z}}       
\newcommand{\rec}{\mathsf{rec}}  
\newcommand{\Set}{\mathsf{Set}}    
\newcommand{\lloop}{\mathsf{loop}}    
\newcommand{\base}{\mathsf{base}}   
\newcommand{\cov}{\mathsf{cov}}   
\newcommand{\suc}{\mathsf{succ}}   
\newcommand{\ua}{\mathsf{ua}}    
\newcommand{\type}{\texttt{type}}       
\newcommand{\id}[1]{\texttt{Id}_{#1}} 
\newcommand{\judge}[3][]{#2\;\vdash_{#1}\;#3}



%%
%% THEOREM LIKE ENVIRONMENT SETTINGS
%%

%\theoremstyle{theorem}
%\newtheorem{theorem}
%\newtheorem{conjecture}[theorem]{Conjecture}
%\theoremstyle{definition}
%\newtheorem{definition}[theorem]{Definition}
%\theoremstyle{remark}
%\newtheorem*{remark}{Remark}

%%%%%%%%%%%%%%%%%%%%%%%%%%%%%%%%%%%%%%%%%%%%%%%%%%%%%%%%%%%%
\begin{document}
%%%%%%%%%%%%%%%%%%%%%%%%%%%%%%%%%%%%%%%%%%%%%%%%%%%%%%%%%%%%

\title{
{\sc Stack representation for pretoposes:\\
  Toward logical schemes}
}
\author{
\emph{The Legacy of Jim Lambek}\\
CLMPS Helsinki\\
4 August 2015
}
\date{
Steve Awodey}

\maketitle

%%%%%%%%%%%%%%%%%%%%%%%%%%%%%%%%%%%%%%%%%%%%%%%%%%%%%%%%%
\begin{frame}{Background}

As a student, I was inspired by a paper of Lambek's (with I.~Moerdijk),``Two sheaf representations for elementary toposes'', 
which applied algebra to logic via category theory. The main result of my own thesis was an extension of this idea. 
\medskip

The basic model for these results was Grothendieck's sheaf representation for commutative
rings. In subsequent work with two PhD students we have pursued this analogy further: 
\medskip

With H.~Forssell, we developed the ``site'' for the sheaf representation of a boolean pretopos as the topological groupoid of models,
resulting in a Stone duality for first-order logic. 
\medskip

With S.~Breiner, we added the ``structure sheaf'' of local pretoposes, to arrive the notion of a ``logical scheme'', which combines the syntax
and semantics of a logical theory into a single object with both aspects. 

\end{frame}
%%%%%%%%%%%%%%%%%%%%%%%%%%%%%%%%%%%%%%%%%%%%%%%%%%%%%%%%%
%%%%%%%%%%%%%%%%%%%%%%%%%%%%%%%%%%%%%%%%%%%%%%%%%%%%%%%%%
\begin{frame}{Outline: Several sheaf representations}


\begin{itemize}

\item Commutative rings: Grothendieck

\item Toposes 1: Lambek and Moerdijk

\item Toposes 2: Lambek

\item Toposes 3: Awodey

\item Boolean algebras: Stone duality

\item Pretoposes 1: A.\ and Forssell

\item Pretoposes 2: A.\ and Breiner

\end{itemize}

\end{frame}
%%%%%%%%%%%%%%%%%%%%%%%%%%%%%%%%%%%%%%%%%%%%%%%%%%%%%%%%%
%%%%%%%%%%%%%%%%%%%%%%%%%%%%%%%%%%%%%%%%%%%%%%%%%%%%%%%%%
\begin{frame}{Grothendieck sheaf representation for commutative rings}

\begin{definition} A  ring (commutative, with unit $1\neq 0$) is called \emph{local} if it has a unique maximal ideal.
Equivalently: $$x+y\ \text{is a unit}\quad\text{implies}\quad x\ \text{is a unit}\ \text{or}\ y\ \text{is a unit}.$$
\end{definition}
%
\begin{theorem}[Grothendieck]
Let $A$ be a ring.  There is a space $X$ with a sheaf of rings $\mathcal{O}$ such that:
\begin{enumerate}
\item for every $p\in X$, the stalk $\mathcal{O}_p$ is a local ring, 
\item for the ring of global sections, we have: $\Gamma(\mathcal{O}) \cong A$.
\end{enumerate}
Thus \emph{every ring is isomorphic to the ring of global sections of a sheaf of local rings}.
\end{theorem}

\end{frame}
%%%%%%%%%%%%%%%%%%%%%%%%%%%%%%%%%%%%%%%%%%%%%%%%%%%%%%%%%
%%%%%%%%%%%%%%%%%%%%%%%%%%%%%%%%%%%%%%%%%%%%%%%%%%%%%%%%%
\begin{frame}{Grothendieck sheaf representation for commutative rings}

The \myemph{space} $X$ is the \emph{prime spectrum} $\mathsf{Spec}(A)$:
 \begin{enumerate}
\item points $p\in \mathsf{Spec}(A)$ are prime ideals $p\subseteq A$,
\item the topology has basic opens of the following form, for all $f\in A$:
 $$B_f = \{ p\in \mathsf{Spec}(A)\ |\ f\not\in p \}.$$
\end{enumerate}
The \myemph{structure sheaf} $\mathcal {O}$ is determined by ``localizing'' $A$ at $f$,
\[
\mathcal{O}(B_f) = [f]^{-1}A
\]
where $A \rightarrow [f]^{-1}A$ freely inverts all of the elements $f, f^2, f^3, \dots$.
\medskip

The \myemph{stalk} $\mathcal {O}_p$ is then the localization of $A$ at $p$,
\[
\mathcal{O}_p = S^{-1}A, 
\]
where $S = A\setminus p$.

\end{frame}
%%%%%%%%%%%%%%%%%%%%%%%%%%%%%%%%%%%%%%%%%%%%%%%%%%%%%%%%%
%%%%%%%%%%%%%%%%%%%%%%%%%%%%%%%%%%%%%%%%%%%%%%%%%%%%%%%%%
\begin{frame}{Grothendieck sheaf representation for commutative rings}

The \myemph{affine scheme} $(\mathsf{Spec}(A), \mathcal {O})$ represents $A$ as a ``ring of continuous functions'' 
\[
f : \mathsf{Spec}(A) \to \mathcal {O}\,,
\]
\myemph{except} that the ring $\mathcal{O}$ is itself ``varying continuously over the space $\mathsf{Spec}(A)$'' (i.e.\ it is a sheaf).  
\medskip

The local ring $\mathcal{O}_p$ has a \myemph{unique maximal ideal}, consisting of ``those functions  $f : \mathsf{Spec}(A) \to \mathcal{O}$ that vanish at $p$''.
\medskip

It is a ``representation'' of $A$ because there is always an injective homomorphism
\[
A \cong \Gamma(\mathcal{O}) \rightarrowtail \prod_{p}\mathcal{O}_p \,.
\]

\begin{corollary}[Sub-direct-product representation]
Every ring $A$ is isomorphic to a \myemph{sub}ring of a \myemph{direct product} of local rings.
\end{corollary}

\end{frame}
%%%%%%%%%%%%%%%%%%%%%%%%%%%%%%%%%%%%%%%%%%%%%%%%%%%%%%%%%
%%%%%%%%%%%%%%%%%%%%%%%%%%%%%%%%%%%%%%%%%%%%%%%%%%%%%%%%%
\begin{frame}{Lambek-Moerdijk sheaf representation for toposes}

\begin{definition} A  (small, elementary)  topos is called \emph{sublocal} if its subterminal lattice $\mathsf{Sub}(1)$ has a unique maximal ideal.
Equivalently, for $x,y\in \mathsf{Sub}(1)$: 
\[
x\vee y = 1\quad\text{implies}\quad x=1\ \text{or}\ y=1\,.
\]
\end{definition}
%
\begin{theorem}[Lambek-Moerdijk 1982]
Let $\E$ be a topos.  There is a space $X$ with a sheaf of toposes $\tilde\E$ such that:
\begin{enumerate}
\item for every $p\in X$, the stalk $\tilde\E_p$ is a sublocal topos, 
\item for the topos of global sections, we have: $\Gamma(\tilde\E) \cong \E$.
\end{enumerate}
Thus \emph{every topos is isomorphic to the topos of global sections of a sheaf of sublocal toposes}.
\end{theorem}

\end{frame}
%%%%%%%%%%%%%%%%%%%%%%%%%%%%%%%%%%%%%%%%%%%%%%%%%%%%%%%%%
%%%%%%%%%%%%%%%%%%%%%%%%%%%%%%%%%%%%%%%%%%%%%%%%%%%%%%%%%
\begin{frame}{Lambek-Moerdijk sheaf representation for toposes}

The \myemph{space} $X$ is the so-called \emph{(sub)spectrum of the topos},  $\mathsf{Spec}(\E)$.\\  
\medskip

It is the prime spectrum of the distributive lattice $\mathsf{Sub}(1)$:
 \begin{enumerate}
\item the points $P\in \mathsf{Spec}(\E)$ are prime ideals $P\subseteq \mathsf{Sub}(1)$,
\item the basic opens have the following form, for all $q\in\mathsf{Sub}(1)$:
$$B_q = \{ P\in \mathsf{Spec}(\E)\ |\ q\not\in P \}.$$
\end{enumerate}

The lattice of all open sets of $\mathsf{Spec}(\E)$ is isomorphic to the ideal completion of $\mathsf{Sub}(1)$,
$$O(\mathsf{Spec}(\E)) = \mathsf{Idl}(\mathsf{Sub}(1))\,.$$

%Equivalently, we could have set:
%\begin{enumerate}
%\item points $F\in Spec(\E)$ are prime filters $F\subseteq \mathsf{Sub}(1)$,
%\item the topology has basic opens of the following form, for $q\in\mathsf{Sub}(1)$:
%$$D_q = \{ F\in Spec(\E)\ \ |\ q\in F \}.$$
%\end{enumerate}
%

\end{frame}
%%%%%%%%%%%%%%%%%%%%%%%%%%%%%%%%%%%%%%%%%%%%%%%%%%%%%%%%%
%%%%%%%%%%%%%%%%%%%%%%%%%%%%%%%%%%%%%%%%%%%%%%%%%%%%%%%%%
\begin{frame}{Lambek-Moerdijk sheaf representation for toposes}


The \myemph{structure sheaf} $\tilde{\E}$ is determined by ``slicing'' $\E$ at $q \in\mathsf{Sub}(1)$,
\[
\tilde{\E}(B_q) = \E/q\,.
\]
This takes the place of localization.  Note that it also ``inverts'' all those elements $p\in \mathsf{Sub}(1)$ with $q\leq p$. 
\bigskip

For the global sections $\Gamma$, we have:
$$\Gamma(\tilde{\E}) \cong \tilde{\E}(B_\top) = \E/1 \cong \E\,.$$
So the topos of global sections of $\tilde{\E}$ is indeed  isomorphic to $\E$.
\end{frame}
%%%%%%%%%%%%%%%%%%%%%%%%%%%%%%%%%%%%%%%%%%%%%%%%%%%%%%%%%
%%%%%%%%%%%%%%%%%%%%%%%%%%%%%%%%%%%%%%%%%%%%%%%%%%%%%%%%%
\begin{frame}{Lambek-Moerdijk sheaf representation for toposes}

The \myemph{stalk} $\tilde{\E}_P$ at a prime ideal $P\in \mathsf{Spec}(\E)$ is the filter-quotient topos,
\[
\tilde{\E}_P = \varinjlim_{q\not\in P} \E/q, 
\]
at the prime \myemph{filter} $\mathsf{Sub}(1)\!\setminus\! P$.  
\medskip

One then has:
\[
\mathsf{Sub}_{\tilde{\E}_P}(1)\cong P\,,
\]
so the stalk topos $\tilde{\E}_P$ is indeed sublocal.

\end{frame}
%%%%%%%%%%%%%%%%%%%%%%%%%%%%%%%%%%%%%%%%%%%%%%%%%%%%%%%%%
%%%%%%%%%%%%%%%%%%%%%%%%%%%%%%%%%%%%%%%%%%%%%%%%%%%%%%%%%
\begin{frame}{Lambek-Moerdijk sheaf representation for toposes}

Again, there is always an injection from the global sections into the product of the stalks,
\[
\E \cong \Gamma(\tilde{\E}) \rightarrowtail \prod_{P\in X}\tilde{\E}_P\,.
\]

\begin{corollary}[Sub-direct-product representation for toposes]
Every topos $\E$ is isomorphic to a \myemph{sub}topos of a \myemph{direct product} of sublocal toposes.
\end{corollary}


\end{frame}
%%%%%%%%%%%%%%%%%%%%%%%%%%%%%%%%%%%%%%%%%%%%%%%%%%%%%%%%%
%%%%%%%%%%%%%%%%%%%%%%%%%%%%%%%%%%%%%%%%%%%%%%%%%%%%%%%%%
\begin{frame}{Lambek-Moerdijk sheaf representation for toposes}

We have the following \myemph{logical interpretation} of the sheaf representation:
%
\begin{itemize}
\item A topos $\E$ is (the term model of) a theory in Intuitionistic Higher-Order Logic.
\item A sublocal topos $\mathcal{S}$ is one that has the \emph{disjunction property}:
\[
\mathcal{S}\vdash p\vee q \qquad\text{iff}\qquad \mathcal{S}\vdash p\  \ \text{or}\  \ \mathcal{S}\vdash q\,,
\]
for all ``propositions'' $p, q$.
\item  The subdirect-product embedding is a logical completeness theorem with respect to such ``semantic'' toposes $\mathcal{S}$.
\item The sheaf representation is a Kripke-style completeness theorem for IHOL, with $\tilde\E$ as a ``sheaf of possible worlds''.
\end{itemize}

\end{frame}
%%%%%%%%%%%%%%%%%%%%%%%%%%%%%%%%%%%%%%%%%%%%%%%%%%%%%%%%%
%%%%%%%%%%%%%%%%%%%%%%%%%%%%%%%%%%%%%%%%%%%%%%%%%%%%%%%%%
\begin{frame}{Lambek's modified sheaf representation for toposes}

But this result is \myemph{not entirely satisfactory}, because we would like the ``semantic worlds'' $\mathcal{S}$ to also have the \emph{existence property}:
\[
\mathcal{S}\vdash (\exists x:A)\varphi(x) \qquad\text{iff}\qquad  \mathcal{S}\vdash \varphi(a)\ \text{for some closed $a:A$}\,,
\] 
(we know that we can prove completeness with respect to such).
\medskip

\begin{definition}
A topos $\mathcal{S}$ is called \myemph{local} if the terminal object $1$ is indecomposable and projective, i.e.\ the global sections functor 
\[
\Gamma = \hom_\mathcal{S}(1, - ) : \mathcal{S} \to \Set
\]
preserves coproducts and epimorphisms.
\end{definition}
\medskip

Note that a local topos has \myemph{both} the disjunction and existence properties.

\end{frame}
%%%%%%%%%%%%%%%%%%%%%%%%%%%%%%%%%%%%%%%%%%%%%%%%%%%%%%%%%
%%%%%%%%%%%%%%%%%%%%%%%%%%%%%%%%%%%%%%%%%%%%%%%%%%%%%%%%%
\begin{frame}{Lambek's modified sheaf representation for toposes}

Lambek gave the following improvement over the sublocal sheaf representation:

\begin{theorem}[Lambek 1989]
Let $\E$ be a topos.  
There is a faithful logical functor $\E\rightarrowtail\mathcal{F}$ 
and a space $X$ with a sheaf of toposes 
$\tilde{\mathcal{F}}$ such that:
\begin{enumerate}
\item for every $p\in X$, the stalk $\tilde{\mathcal{F}}_p$ is a \myemph{local} topos, 
\item for the topos of global sections, we have: $\Gamma(\tilde{\mathcal{F}}) \cong \mathcal{F}$.
\end{enumerate}
Thus every topos is a \myemph{subtopos} of one that is isomorphic to the topos of global sections of a sheaf of \myemph{local} toposes.  
\end{theorem}
\medskip

This suffices for a \emph{sub-direct-product representation} into \myemph{local} toposes, and therefore gives the desired \emph{logical completeness} with respect to \myemph{local} toposes.  

But conceptually it is still not entirely satisfactory.

\end{frame}
%%%%%%%%%%%%%%%%%%%%%%%%%%%%%%%%%%%%%%%%%%%%%%%%%%%%%%%%%
%%%%%%%%%%%%%%%%%%%%%%%%%%%%%%%%%%%%%%%%%%%%%%%%%%%%%%%%%
\begin{frame}{Local sheaf representation for toposes}

In my thesis, I proved:

\begin{theorem}[A. 1998]
Let $\E$ be a topos.  
There is a space $X$ with a sheaf of toposes $\tilde{\E}$ such that:
\begin{enumerate}
\item for every $p\in X$, the stalk $\tilde{\E}_p$ is a \myemph{local} topos, 
\item for the topos of global sections, we have: $\Gamma(\tilde{\E}) \cong \E$.
\end{enumerate}
Thus every topos is isomorphic to the global sections of a sheaf of \myemph{local} toposes.  
\end{theorem}
\medskip

As before, this gives a \emph{sub-direct-product representation},
\[
\E \rightarrowtail \prod_{p}\mathcal{S}_p
\]
into a product of local toposes $\mathcal{S}_p$, and therefore implies the desired \emph{logical completeness} of IHOL with respect to local toposes.  
\end{frame}
%%%%%%%%%%%%%%%%%%%%%%%%%%%%%%%%%%%%%%%%%%%%%%%%%%%%%%%%%
%%%%%%%%%%%%%%%%%%%%%%%%%%%%%%%%%%%%%%%%%%%%%%%%%%%%%%%%%
\begin{frame}{Local sheaf representation for toposes}

The stronger result also gives better ``Kripke semantics'' for IHOL, since the ``sheaf of possible worlds'' now has \myemph{local} stalks.
\medskip

For \myemph{classical} higher-order logic, more can be said:

\begin{lemma}
Every local \myemph{boolean} topos is well-pointed, i.e.\ the global sections functor,
\[
\Gamma = \hom_\mathcal{S}(1, - ) : \mathcal{S} \to \Set
\]
is faithful.
\end{lemma}

A well-pointed topos is essentially a model of set theory.  

\begin{corollary}
Every boolean topos is isomorphic to the global sections of a sheaf of \myemph{well-pointed} toposes.  
\end{corollary}

\end{frame}
%%%%%%%%%%%%%%%%%%%%%%%%%%%%%%%%%%%%%%%%%%%%%%%%%%%%%%%%%
%%%%%%%%%%%%%%%%%%%%%%%%%%%%%%%%%%%%%%%%%%%%%%%%%%%%%%%%%
\begin{frame}{Local sheaf representation for toposes}

For boolean toposes, we therefore have the  representation, 
\[
\mathcal{B} \rightarrowtail \prod_{p}\mathcal{S}_p 
\]
as sub-direct-product of \emph{well-pointed} toposes $\mathcal{S}_p$, along with its logical counterpart:
%
\begin{corollary}
Classical HOL is complete with respect to models in well-pointed toposes.
\end{corollary}
\medskip

These are \myemph{standard} models of classical HOL, taken in varying (``non-standard'') models of set theory.

\end{frame}
%%%%%%%%%%%%%%%%%%%%%%%%%%%%%%%%%%%%%%%%%%%%%%%%%%%%%%%%%
%%%%%%%%%%%%%%%%%%%%%%%%%%%%%%%%%%%%%%%%%%%%%%%%%%%%%%%%%
\begin{frame}{Local sheaf representation for toposes}


Taking the global sections $\Gamma:\mathcal{S}_p \rightarrowtail \Set$ of each such well-pointed model then embeds any boolean topos $\mathcal{B}$ into a power of $\Set$:
\[
\mathcal{B} \rightarrowtail \prod_{p}\mathcal{S}_p \rightarrowtail \prod_{p}\Set_p \cong \Set^X\,,
\]
The various composites $\mathcal{B} \rightarrow \mathcal{S}_p \rightarrowtail \Set$ are Henkin style, ``non-standard'' models of HOL in $\Set$.

\begin{corollary}
Classical HOL is complete with respect to Henkin models in $\Set$.
\end{corollary}
\medskip

These Henkin models can be taken as the points of the space $X_\E$ for the sheaf representation.

\end{frame}
%%%%%%%%%%%%%%%%%%%%%%%%%%%%%%%%%%%%%%%%%%%%%%%%%%%%%%%%%
%%%%%%%%%%%%%%%%%%%%%%%%%%%%%%%%%%%%%%%%%%%%%%%%%%%%%%%%%
\begin{frame}{Local sheaf representation for toposes}
To define the \myemph{space $X_\E$ of models}:
\medskip

In the \myemph{sublocal} case, the points were \emph{prime ideals} $p\subseteq\mathsf{Sub}(1)$.
These correspond exactly to \emph{lattice homomorphisms} $$p: \mathsf{Sub}_{\E}(1)\to \mathbf{2}\,.$$

For the \myemph{local} case, we instead take \emph{coherent functors} $$P: \E\to\Set\,.$$
These correspond exactly to Henkin models of (the theory represented by) $\E$.
\medskip

The \myemph{topology} is given (roughly speaking) by basic open sets of the following form, for all formulas $\varphi$:
\[
V_\varphi = \{ P\ |\ P\models \varphi \}
\]

\end{frame}
%%%%%%%%%%%%%%%%%%%%%%%%%%%%%%%%%%%%%%%%%%%%%%%%%%%%%%%%%
%%%%%%%%%%%%%%%%%%%%%%%%%%%%%%%%%%%%%%%%%%%%%%%%%%%%%%%%%
\begin{frame}{Local sheaf representation for toposes}

The \myemph{structure sheaf} $\tilde\E$ is  first defined as a \myemph{stack} on $\E$ by ``slicing'',
\[
\tilde\E(A)\ =\ \E/A\,.
\] 
The stack is first strictified to a \myemph{sheaf}, and then transferred from $\E$ to the space $X_\E$ of models using a topos-theoretic covering theorem due to Butz and Moerdijk.
\medskip

For the \myemph{global sections} $\Gamma$, we then have:
$$\Gamma(\tilde{\E})\ \simeq\ \E/1\ \cong\ \E\,.$$

And for the \myemph{stalks} $\tilde{\E}_P$ we have the colimit,
\[
\tilde{\E}_P\ =\ \varinjlim_{A\,\in\int\!{P}} \E/A, 
\]
where the (filtered!) category of elements $\int\!{P}$ of the Henkin model $P$ takes the place of the prime filter.  

\end{frame}
%%%%%%%%%%%%%%%%%%%%%%%%%%%%%%%%%%%%%%%%%%%%%%%%%%%%%%%%%
%%%%%%%%%%%%%%%%%%%%%%%%%%%%%%%%%%%%%%%%%%%%%%%%%%%%%%%%%
\begin{frame}{Boolean algebras and Stone duality}

The results for toposes suggest an analogous treatment for \myemph{pretoposes} which would be somewhat better, 
because the models involved would all be \myemph{standard} ones, rather than Henkin style, non-standard models.
\medskip

Moreover, we then recognize the analogy to Stone duality for Boolean algebras.  
From the logical point of view, we have a new \myemph{Stone duality for first-order logic}, with the classical theory as the propositional case.
\medskip

\end{frame}
%%%%%%%%%%%%%%%%%%%%%%%%%%%%%%%%%%%%%%%%%%%%%%%%%%%%%%%%%
%%%%%%%%%%%%%%%%%%%%%%%%%%%%%%%%%%%%%%%%%%%%%%%%%%%%%%%%%
\begin{frame}{Boolean algebras and Stone duality}


Recall that for a boolean algebra $B$ we have the Stone space $\mathsf{Spec}(B)$,  defined exactly as for the subterminal lattice $\mathsf{Sub}_{\E}(1)$ of a topos $\E$ (i.e.\ the prime spectrum).  We can represent the \myemph{points} $p\in \mathsf{Spec}(B)$ as boolean homomorphisms,
\[
p : B\to \mathbf{2}\,.
\]

We can recover $B$ from the space $\mathsf{Spec}(B)$ as the \myemph{clopen subsets}, which are represented by continuous maps,
\[
f : \mathsf{Spec}(B)\to \mathbf{2}\,,
\]
where $\mathbf{2}$ is given the discrete topology. (This is just a \myemph{constant} sheaf representation!)


\end{frame}
%%%%%%%%%%%%%%%%%%%%%%%%%%%%%%%%%%%%%%%%%%%%%%%%%%%%%%%%%
%%%%%%%%%%%%%%%%%%%%%%%%%%%%%%%%%%%%%%%%%%%%%%%%%%%%%%%%%
\begin{frame}{Boolean algebras and Stone duality}

There is a contravariant equivalence of categories,
\[
\xymatrix{ 
\mathbf{Bool}  \ar@/{}^{1pc}/[rr]^{\mathsf{Spec}}     &&  \mathbf{Stone}^{\mathsf{op}}  \ar@/{}^{1pc}/[ll]^{\mathsf{Clop}}  \\
} 
\]
The functors are given just by homming into $\mathbf{2}$.
\medskip

Logically, a Boolean algebra is (the Lindenbaum-Tarski algebra of) a \myemph{theory in propositional logic}, and a boolean homomorphism $B\to \mathbf{2}$ is a \myemph{model}, i.e.\ a truth-valuation.
\medskip

We shall generalize this situation by replacing Boolean algebras with (Boolean) pretoposes, representing first-order theories, and replacing $\mathbf{2}$-valued models with $\mathbf{Set}$-valued models.

\end{frame}
%%%%%%%%%%%%%%%%%%%%%%%%%%%%%%%%%%%%%%%%%%%%%%%%%%%%%%%%%
%%%%%%%%%%%%%%%%%%%%%%%%%%%%%%%%%%%%%%%%%%%%%%%%%%%%%%%%%
\begin{frame}{Stone duality for pretoposes (A.-Forssell)}

The generalization to Boolean pretoposes works like this:
\medskip

\begin{center}
\begin{tabular}{c|c}
Boolean algebra $B$ & Boolean pretopos $\mathcal{B}$ \\
propositional theory & first-order theory \\
\hline homomorphism & pretopos functor\\
$B\to \mathbf{2}$ &  $\mathcal{B}\to \mathbf{Set}$ \\
truth-valuation & elementary model \\
\hline topological space & topological groupoid \\
 $\mathsf{Spec}(B)$ & $\mathbf{Spec}(\mathcal{B})$ \\
of all valuations & of all models and isos \\
\hline continuous function & coherent functor \\
$\mathsf{Spec}(B) \to \mathbf{2}$  & $\mathbf{Spec}(\mathcal{B}) \to \mathbf{Set}$\\
clopen set & coherent sheaf \\
\end{tabular}
\end{center}

\end{frame}
%%%%%%%%%%%%%%%%%%%%%%%%%%%%%%%%%%%%%%%%%%%%%%%%%%%%%%%%%
%%%%%%%%%%%%%%%%%%%%%%%%%%%%%%%%%%%%%%%%%%%%%%%%%%%%%%%%%
\begin{frame}{Stone duality for pretoposes (A.-Forssell)}


\begin{theorem}[A.-Forssell 2008]
There is a contravariant \myemph{adjunction},
\[
\xymatrix{ 
\mathbf{BPreTop}  \ar@/{}^{1pc}/[rr]^{\mathsf{Spec}}     &&  \mathbf{StoneTopGpd}^{\mathsf{op}}  \ar@/{}^{1pc}/[ll]^{\mathsf{Coh}} \,,
} 
\]
in which the functors are given by homming into $\mathbf{Set}$.
\end{theorem}

The spectrum $\mathbf{Spec}(\mathcal{B})$ of a Boolean pretopos $\mathcal{B}$ is the groupoid of models and isos, topologized by ``satisfaction of formulas''.
\medskip

Recovering $\mathcal{B}$ from $\mathbf{Spec}(\mathcal{B})$ amounts to recovering an elementary theory from its models.  This is done by taking the coherent, equivariant sheaves on $\mathbf{Spec}(\mathcal{B})$, using results from topos theory.

\end{frame}
%%%%%%%%%%%%%%%%%%%%%%%%%%%%%%%%%%%%%%%%%%%%%%%%%%%%%%%%%
%%%%%%%%%%%%%%%%%%%%%%%%%%%%%%%%%%%%%%%%%%%%%%%%%%%%%%%%%
\begin{frame}{Stone duality for pretoposes (A.-Forssell)}

Makkai has found a related \myemph{equivalence}:
\[
\xymatrix{ 
\mathbf{BPreTop}  \ar@/{}^{1pc}/[rr]     & \simeq &  \mathbf{UltraGpd}^{\mathsf{op}}  \ar@/{}^{1pc}/[ll] \\
} 
\]
We replace Makkai's \myemph{ultraproduct} structure on the groupoids of models by a Stone-style \myemph{logical topology}.
\medskip

For us, however, the ``semantic'' functor,
\[
\mathsf{Spec} : \mathbf{BPreTop} \longrightarrow \mathbf{StoneTopGpd}^{\mathsf{op}}
\]
 is \myemph{not full}: there are continuous functors between the groupoids of models that do not come from a ``translation of theories''.
\medskip

Compare the case of \myemph{rings} $A, B$:  a continuous function $$f : \mathsf{Spec}(B) \to \mathsf{Spec}(A)$$ need not come from a ring homomorphism $h : A\to B$. 

\end{frame}
%%%%%%%%%%%%%%%%%%%%%%%%%%%%%%%%%%%%%%%%%%%%%%%%%%%%%%%%%
%%%%%%%%%%%%%%%%%%%%%%%%%%%%%%%%%%%%%%%%%%%%%%%%%%%%%%%%%
\begin{frame}{Sheaf representation for pretoposes (A.-Breiner)}

As for rings and affine schemes, we can equip the spectrum $\mathbf{Spec}(\mathcal{B})$ of the pretopos $\mathcal{B}$ with a ``structure sheaf'' $\tilde{\mathcal{B}}$,\\
  defined as in the sheaf representation for toposes: 
\begin{itemize}
\item start with the presheaf of categories $\tilde{\mathcal{B}} : \mathcal{B}^{\mathrm{op}}\to\mathbf{Cat}$ with,
\[
\tilde{\mathcal{B}}(X) \cong \mathcal{B}/X\,,
\]
for all $X\in \B$. This is a \myemph{stack} because $\mathcal{B}$ is a pretopos.
 
\item Strictify to get a sheaf of categories on $\mathcal{B}$. 

\item Use the equivalence of toposes,
\[
\mathbf{Sh}(\B)\ \simeq\ \mathbf{Sh}_{\mathsf{eq}}(\mathbf{Spec}(\mathcal{B}))
\]
$\mathbf{Spec}(\mathcal{B})$ is determined so that this equivalence holds.

\item Move $\tilde{\mathcal{B}}$ along this equivalence in order to get an equivariant sheaf on $\mathbf{Spec}(\mathcal{B})$.
\end{itemize}

\end{frame}
%%%%%%%%%%%%%%%%%%%%%%%%%%%%%%%%%%%%%%%%%%%%%%%%%%%%%%%%%
%%%%%%%%%%%%%%%%%%%%%%%%%%%%%%%%%%%%%%%%%%%%%%%%%%%%%%%%%
\begin{frame}{Sheaf representation for pretoposes (A.-Breiner)}

The transported $\tilde{\mathcal{B}}$  is an equivariant sheaf of local, boolean pretoposes on the topological groupoid $\mathbf{Spec}(\mathcal{B})$.

\begin{itemize}
\item Logically, $\tilde{\mathcal{B}}$ is a sheaf of ``local theories'' on the groupoid of models, equipped with the logical topology.  

\item As before, $\tilde{\mathcal{B}}$ has global sections $\Gamma\tilde{\mathcal{B}} \simeq \mathcal{B}$.  So the original ``theory'' $\mathcal{B}$ is the ``theory of all models''.

\item The stalk $\tilde{\mathcal{B}}_P$ at a model $P : \mathcal{B} \to \Set$ is a well-pointed pretopos: it is the ``elementary diagram'' of the model $P$.
\medskip

\item The global sections functors $\Gamma_P : \tilde{\mathcal{B}}_P \rightarrowtail \Set$ are faithful \myemph{pretopos} morphisms, i.e.\ ``complete models''.

\end{itemize}
\end{frame}
%%%%%%%%%%%%%%%%%%%%%%%%%%%%%%%%%%%%%%%%%%%%%%%%%%%%%%%%%
%%%%%%%%%%%%%%%%%%%%%%%%%%%%%%%%%%%%%%%%%%%%%%%%%%%%%%%%%
\begin{frame}{Sheaf representation for pretoposes (A.-Breiner)}

In sum, we have the following:

\begin{theorem}[A.-Breiner 2013]
Let $\B$ be a boolean pretopos.  
There is a topological groupoid $G$ with an equivariant sheaf of pretoposes $\tilde{\B}$ such that:
\begin{enumerate}
\item for every $x\in G$, the stalk $\tilde{\B}_x$ is a well-pointed pretopos, 
\item for the pretopos of global sections, we have: $\Gamma(\tilde{\B}) \cong \B$.
\end{enumerate}
Thus every Boolean pretopos is isomorphic to the global sections of a sheaf of well-pointed pretoposes.  
\end{theorem}
\medskip

There is an analogous result for non-Boolean case, with local pretoposes in place of well-pointed ones in the stalks.

\end{frame}
%%%%%%%%%%%%%%%%%%%%%%%%%%%%%%%%%%%%%%%%%%%%%%%%%%%%%%%%%
%%%%%%%%%%%%%%%%%%%%%%%%%%%%%%%%%%%%%%%%%%%%%%%%%%%%%%%%%
\begin{frame}{Sheaf representation for pretoposes (A.-Breiner)}

The resulting sub-direct-product representation $\mathcal{B}\rightarrowtail \prod_{x}\tilde{\mathcal{B}}_x$ yields:\\[1ex]

\begin{corollary}[G\"odel completeness theorem] 
There is a pretopos embedding,
\[
\mathcal{B}\rightarrowtail \prod_{x\in G}\tilde{\mathcal{B}}_{x} \rightarrowtail \prod_{x\in G}\Set \simeq \Set^{|G|}\,,
\]
with $|G|$ the set of points of the topological groupoid $G = \mathbf{Spec}(\mathcal{B})$ of models.
\end{corollary}
\medskip

Other consequences of the sheaf representation include:

\begin{itemize}
\item Conceptual completeness (Makkai-Reyes)
\item Beth definability (Makkai)
\end{itemize}


\end{frame}
%%%%%%%%%%%%%%%%%%%%%%%%%%%%%%%%%%%%%%%%%%%%%%%%%%%%%%%%%
%%%%%%%%%%%%%%%%%%%%%%%%%%%%%%%%%%%%%%%%%%%%%%%%%%%%%%%%%
\begin{frame}{Logical schemes}

For a Boolean pretopos $\B$ , call the pair $$(\mathbf{Spec}(\mathcal{B}), \tilde{\mathcal{B}})$$ an affine \myemph{logical scheme}.  \\

A morphism of logical schemes
\[
(h, \tilde{h}) : (\mathbf{Spec}(\mathcal{B}), \tilde{\mathcal{B}}) \to (\mathbf{Spec}(\mathcal{A}), \tilde{\mathcal{A}})
\]
consists of a continuous groupoid homomorphism 
\[
h : \mathbf{Spec}(\mathcal{B}) \to \mathbf{Spec}(\mathcal{A}),
\]
together with a pretopos functor 
\[
\tilde{h} : \tilde{\mathcal{A}} \to h_*\tilde{\mathcal{B}}
\]
over $\mathbf{Spec}(\mathcal{A})$.

\end{frame}
%%%%%%%%%%%%%%%%%%%%%%%%%%%%%%%%%%%%%%%%%%%%%%%%%%%%%%%%%
%%%%%%%%%%%%%%%%%%%%%%%%%%%%%%%%%%%%%%%%%%%%%%%%%%%%%%%%%
\begin{frame}{Logical schemes}

\begin{theorem}[A.-Breiner 2012]
Every pretopos functor $\mathcal{A} \to \mathcal{B}$ induces a morphism of the associated affine logical schemes.  Moreover, the  functor
\[
\mathsf{Spec} : \mathbf{BPreTop} \longrightarrow \mathbf{AffLScheme}^{\mathsf{op}}
\]
is \myemph{full}: every map of schemes comes from a map of pretoposes.
\end{theorem}

\begin{corollary}
There is a contravariant equivalence:
\[
\mathbf{BPreTop} \ \simeq\ \mathbf{AffLScheme}^{\mathsf{op}}
\]
between Boolean pretoposes and affine logical schemes.
\end{corollary}

\end{frame}
%%%%%%%%%%%%%%%%%%%%%%%%%%%%%%%%%%%%%%%%%%%%%%%%%%%%%%%%%
%%%%%%%%%%%%%%%%%%%%%%%%%%%%%%%%%%%%%%%%%%%%%%%%%%%%%%%%%
\begin{frame}{References}

\begin{itemize}
\item Lambek, J.\ and Moerdijk, I.,\\
 ``Two sheaf representations of elementary toposes'',\\
in: A.S.\ Troelstra, D.\ van Dalen (Eds.), \emph{Brouwer Centenary Symposium}, North-Holland, Amsterdam, 1982.
\item Lambek, J., ``On the sheaf of possible worlds'',\\
 in: Adamek, Mac Lane (Eds.), \emph{Categorical Topology}, World Scientific, Singapore, 1989.
\item Awodey, S., ``Sheaf representation for topoi'',\\
 \emph{J.\ Pure Appl.\ Algebra}, 145, pp.~107--121, 2000.
\item Awodey, S. and Forssell, H., ``First-order logical duality'',\\
 \emph{Ann.\ Pure and Appl.\ Logic}, 164(3), pp.~319--348, 2013.
\item Breiner, S., \emph{Scheme representation for first-order logic},\\
 Ph.D. thesis, CMU, 2013.
\end{itemize}

\end{frame}
%%%%%%%%%%%%%%%%%%%%%%%%%%%%%%%%%%%%%%%%%%%%%%%%%%%%%%%%%








%%
\end{document}
%%
