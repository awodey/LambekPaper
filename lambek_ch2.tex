
\documentclass[lambek.tex]{subfiles}


\begin{document}

\section{Toward logical duality}

The results for toposes suggest an analogous treatment for \myemph{pretoposes} which would be somewhat better, 
because the models involved would all be \myemph{standard} ones, rather than Henkin style, non-standard models.
\medskip

We then have the possibility of a \myemph{logical duality theory} analogous to Grothendieck's duality for schemes and commutative rings, with  
the sheaf representation playing the role of a \myemph{logical structure sheaf}.
\medskip

This can be seen as a generalization of classical Stone duality for Boolean algebras (= Boolean rings):
from a logical point of view, we have a \myemph{Stone duality for first-order logic}, 
with the classical theory for Boolean algebras appearing as the propositional case.

%\end{frame}
%%%%%%%%%%%%%%%%%%%%%%%%%%%%%%%%%%%%%%%%%%%%%%%%%%%%%%%%%
%%%%%%%%%%%%%%%%%%%%%%%%%%%%%%%%%%%%%%%%%%%%%%%%%%%%%%%%%
\section{Boolean algebras and Stone duality}


Recall that for a boolean algebra $B$ we have the Stone space $\mathsf{Spec}(B)$,  defined exactly as for the subterminal lattice $\mathsf{Sub}_{\E}(1)$ of a topos $\E$ (i.e.\ the prime spectrum).  We can represent the \myemph{points} $p\in \mathsf{Spec}(B)$ as boolean homomorphisms,
\[
p : B\to \mathbf{2}\,.
\]

We can recover $B$ from the space $\mathsf{Spec}(B)$ as the \myemph{clopen subsets}, which are represented by continuous maps,
\[
f : \mathsf{Spec}(B)\to \mathbf{2}\,,
\]
where $\mathbf{2}$ is given the discrete topology. (This is just a \myemph{constant} sheaf representation!)


%\end{frame}
%%%%%%%%%%%%%%%%%%%%%%%%%%%%%%%%%%%%%%%%%%%%%%%%%%%%%%%%%
%%%%%%%%%%%%%%%%%%%%%%%%%%%%%%%%%%%%%%%%%%%%%%%%%%%%%%%%%
%\section{Boolean algebras and Stone duality}

There is a contravariant equivalence of categories,
\[
\xymatrix{ 
\mathbf{Bool}  \ar@/{}^{1pc}/[rr]^{\mathsf{Spec}}     &&  \mathbf{Stone}^{\mathsf{op}}  \ar@/{}^{1pc}/[ll]^{\mathsf{Clop}}  \\
} 
\]
The functors are given just by homming into $\mathbf{2}$.
\medskip

Logically, a Boolean algebra is (the Lindenbaum-Tarski algebra of) a \myemph{theory in propositional logic}, and a boolean homomorphism $B\to \mathbf{2}$ is a \myemph{model}, i.e.\ a truth-valuation.
\medskip

We shall generalize this situation by replacing Boolean algebras with (Boolean) pretoposes, representing \myemph{first-order} logical theories, and replacing $\mathbf{2}$-valued models with $\mathbf{Set}$-valued models.

%\end{frame}
%%%%%%%%%%%%%%%%%%%%%%%%%%%%%%%%%%%%%%%%%%%%%%%%%%%%%%%%%
%%%%%%%%%%%%%%%%%%%%%%%%%%%%%%%%%%%%%%%%%%%%%%%%%%%%%%%%%
\section{Lawvere duality}

Consider first the simpler case of equational logic, rather than full first-order logic:
\begin{itemize}

\item In place of a Boolean algebra representing a ``propositional theory'', we have a category $\mathbb{C}_T$ with finite products, representing an algebraic theory $T$ (such as groups). 

\item $\mathbb{C}_T$ may be taken to be the \emph{dual}  of the category $T\mathsf{Alg}_{fg}$ of all finitely generated free algebras and algebra homomorphisms between them.


\item The general $T$-algebras then correspond to FP-functors $\mathbb{C}_T \to \Set$, where the category $\Set$ now plays the role of the ``ring of values'', in place of the Boolean algebra $\{ 0 \leq 1\}$ of ``truth values''. 
\end{itemize}


\begin{theorem}[Lawvere 1963]
There is an equivalence of categories,
\[
T\mathsf{Alg}\ \simeq\ FP(T\mathsf{Alg}_{fg}^\mathsf{op}, \Set)\,.
\]
\end{theorem}

%\end{frame}
%%%%%%%%%%%%%%%%%%%%%%%%%%%%%%%%%%%%%%%%%%%%%%%%%%%%%%%%%
%%%%%%%%%%%%%%%%%%%%%%%%%%%%%%%%%%%%%%%%%%%%%%%%%%%%%%%%%
\section{Lawvere duality and others}

The following dualities are determined in essentially the same way:

\begin{center}
\begin{tabular}{c|c|c}
Lawvere & equational logic & finite product categories \\
& operations and equations & \\
\hline
Gabriel-Ulmer$^{\strut}$ & relational logic & finite limit categories \\
& operations, $=$, relations &  \\
\hline
Makkai$^{\strut}$ & regular logic & regular categories \\
& operations, $=$, relations, $\exists$ & \\
\end{tabular}
\end{center}


- In each case the ``algebraic'' side consists of a structured category representing the logical theory, and the ``spatial'' side consists of structure-preserving functors into $\Set$, which are the models.
\smallskip

- Recovering the ``algebra'' from the ``space'' (the theory from the models) requires a Stone-like represetation/completeness theorem.
\smallskip

- But the situation with the further logical operations $\forall$, $\Rightarrow$, and $\neg$ is somewhat different, because they have a contravariant aspect that cannot be recovered from homomorphisms of models.

%\end{frame}
%%%%%%%%%%%%%%%%%%%%%%%%%%%%%%%%%%%%%%%%%%%%%%%%%%%%%%%%%
\section{Stone duality for pretoposes (A.-Forssell)}

The further generalization of Stone duality to Boolean pretoposes works like this:
\medskip

\begin{center}
\begin{tabular}{c|c}
Boolean algebra $B$ & Boolean pretopos $\mathcal{B}$ \\
propositional theory & first-order theory \\
\hline homomorphism & pretopos functor\\
$B\to \mathbf{2}$ &  $\mathcal{B}\to \mathbf{Set}$ \\
truth-valuation & elementary model \\
\hline topological space & topological \myemph{groupoid} \\
 $\mathsf{Spec}(B)$ & $\mathbf{Spec}(\mathcal{B})$ \\
of all valuations & of all models and isos \\
\hline continuous function & coherent functor \\
$\mathsf{Spec}(B) \to \mathbf{2}$  & $\mathbf{Spec}(\mathcal{B}) \to \mathbf{Set}$\\
clopen set & coherent sheaf 
\end{tabular}
\end{center}

%\end{frame}
%%%%%%%%%%%%%%%%%%%%%%%%%%%%%%%%%%%%%%%%%%%%%%%%%%%%%%%%%
%%%%%%%%%%%%%%%%%%%%%%%%%%%%%%%%%%%%%%%%%%%%%%%%%%%%%%%%%
%\section{Stone duality for pretoposes (A.-Forssell)}


\begin{theorem}[A.-Forssell 2008]
There is a contravariant \myemph{adjunction},
\[
\xymatrix{ 
\mathbf{BPreTop}  \ar@/{}^{1pc}/[rr]^{\mathsf{Spec}}     &&  \mathbf{StoneTopGpd}^{\mathsf{op}}  \ar@/{}^{1pc}/[ll]^{\mathsf{Coh}} \,,
} 
\]
in which the functors are given by homming into $\mathbf{Set}$.
\end{theorem}

The spectrum $\mathbf{Spec}(\mathcal{B})$ of a Boolean pretopos $\mathcal{B}$ is the groupoid of models and isos, topologized by ``satisfaction of formulas''.
\medskip

Recovering $\mathcal{B}$ from $\mathbf{Spec}(\mathcal{B})$ amounts to recovering an elementary theory from its models.  This is done by taking the coherent, equivariant sheaves on $\mathbf{Spec}(\mathcal{B})$, using results from topos theory.

%\end{frame}
%%%%%%%%%%%%%%%%%%%%%%%%%%%%%%%%%%%%%%%%%%%%%%%%%%%%%%%%%
%%%%%%%%%%%%%%%%%%%%%%%%%%%%%%%%%%%%%%%%%%%%%%%%%%%%%%%%%
%\section{Stone duality for pretoposes (A.-Forssell)}

Makkai has found a related \myemph{equivalence}:
\[
\xymatrix{ 
\mathbf{BPreTop}  \ar@/{}^{1pc}/[rr]     & \simeq &  \mathbf{UltraGpd}^{\mathsf{op}}  \ar@/{}^{1pc}/[ll] \\
} 
\]
We replace Makkai's \myemph{ultraproduct} structure on the groupoids of models by a Stone-Zariski type \myemph{logical topology}.
\medskip

For us, however, the ``semantic'' functor,
\[
\mathsf{Spec} : \mathbf{BPreTop} \longrightarrow \mathbf{StoneTopGpd}^{\mathsf{op}}
\]
 is \myemph{not full}: there are continuous functors between the groupoids of models that do not come from a ``translation of theories''.
\medskip

Compare the case of commutative rings $A, B$:  a continuous function $$f : \mathsf{Spec}(B) \to \mathsf{Spec}(A)$$ need not come from a ring homomorphism $h : A\to B$. 





\end{document}